\documentclass[a4paper, 12pt]{article}
\usepackage{ctex}
\usepackage{listings}
\usepackage{graphicx}
\usepackage[bookmarks=true]{hyperref}
\lstset{language=Matlab, %
    %basicstyle=\color{red}, 
    breaklines=true, %
    morekeywords={matlab2tikz}, 
    keywordstyle=\color{blue}, %
    morekeywords=[2]{1},  keywordstyle=[2]{\color{black}}, 
    identifierstyle=\color{black}, %
    stringstyle=\color{mylilas}, 
    commentstyle=\color{mygreen}, %
    showstringspaces=false, %without this there will be a symbol in the places where there is a space
    numbers=left, %
    numberstyle={\tiny \color{black}}, % size of the numbers
    numbersep=9pt,  % this defines how far the numbers are from the text
    emph=[1]{for, end, break}, emphstyle=[1]\color{red},  %some words to emphasise
    %emph=[2]{word1, word2},  emphstyle=[2]{style},     
}
% \thispagestyle{headings}
\begin{document}
  \title{数字通信实验四~调制解调}
  \author{PB16061024 陈进泽}
  \date{\today}
  \maketitle
  \newpage
  \section{实验目的}
  \begin{itemize}
    \item 掌握数字频带传输系统调制解调的仿真过程
    \item 掌握数字频带传输系统误码率仿真分析方法
  \end{itemize}

  \section{实验原理}
  以BPSK为例,   数字频带传输过程如下:

  \includegraphics[width=\textwidth]{BPSK.png}

  假定:信道为加性高斯白噪声信道,   其均值为0、方差为$\sigma^2$,   采用矩形成形;发射端BPSK调制信号为:
  \begin{displaymath}
    s(t)=\left\{
      \begin{array}{cc}
        A \cos{2\pi f_c t} & b_k=''1''\\
        -A \cos{2\pi f_c t} & b_k=''0''
      \end{array}
      \qquad kT\leq t <(k+1)T
    \right.
  \end{displaymath}
  经信道传输,   接受端输入信号为:
  \begin{displaymath}
    y_d(t)=s(t)+n(t)
  \end{displaymath}
  经相干解调,   匹配滤波,   定时恢复后输出:
  \begin{displaymath}
    x_k=\left\{
      \begin{array}{cc}
        A+n_k & b_k=''1''\\
        -A+n_k & b_k=''0''
      \end{array}
      \right.
  \end{displaymath}

  当1,   0独立等概出现时,   BPSK系统的最佳判决门限电平$U_{d}^{*}=0$。故判决规则为在取样时刻的判决值大于0,   判1,   小于0,   判0
  \section{实验内容}
    \subsection{分别编写BPSK与QPSK调制解调系统的Matlab仿真程序}
      \subsubsection{BPSK源代码}
        \lstinputlisting{BPSK.m}
        为了方便绘制误码率曲线时调用采用了函数形式。以下解释代码中内容:
      \subsubsection{BPSK代码解释}
        \begin{description}
          \item 代码15行生成N个随机信源符号
          \item 代码17到20行产生相应的基带信号,   根升余弦滤波器滚降系数为0.5
          \item 代码22行进行信源的调制
          \item 代码24行加噪声
          \item 代码26到28行进行信号的解调,   匹配滤波及采样判决
        \end{description}
        其中因为发送滤波器的影响,   需根据SNR对噪声进行调整。这里采用了等效为码元功率的变化的方式等效了其影响,   具体体现在代码12行
      \subsubsection{QPSK源代码}
        \lstinputlisting{QPSK.m}
        同BPSK一致采用了函数形式表达
      \subsubsection{QPSK代码解释}
        \begin{description}
          \item 代码将发送信号分两路进行发送,   并分两路解调
          \item 代码31行判决接受到的码元各位是否与发送码元一致
          \item 代码32行判决若所有位的码元均相等,   则接受成功
        \end{description}
    \subsection{绘制BPSK与QPSK调制下的误码率与新噪比曲线图,   并与理论曲线进行对比}
      \subsubsection{BPSK}
        \lstinputlisting{PSKEVAL.m}
      \subsubsection{实验效果}
        \includegraphics[width=\textwidth]{BPSKEVAL.png}

        由图可见,   仿真曲线与实际曲线十分相似,   说明理论正确,   解调方式也无误
      \subsubsection{QPSK}
        \lstinputlisting{QPSKEVAL.m}
      \subsubsection{实验效果}
        \includegraphics[width=\textwidth]{QPSKEVAL.png}

        如图,   由QPSK函数计算的是误码率,   上方曲线代表理论误码率曲线,   为$1-(1-\frac{1}{2}erfc(\sqrt{snr})^2$。下方曲线为理论误比特率曲线,   与BPSK情况下的误码率曲线一致,   且在对数坐标下几乎只差个平移,   说明两者为倍数关系,   与理论值$P_e\approx erfc(\sqrt{snr})^2=2P_b$一致,   说明调制与解调方法没错,   且与理论一致
    \subsection{几点说明}
      \begin{enumerate}
        \item 由于实际随机生成序列长度有限,   当误码率足够小时可以想到错误的点的个数也足够小,   故计算获得的误码率偏差较大。这点在仿真图上也有体现,   当SNR足够大时误码率足够小,   故理论曲线与仿真曲线偏差较大,   而在小SNR处两者吻合的很好
        \item 由于filter进行的是卷积操作,   原信号会有所偏移。且$rcosdesign$函数生成的函数长度为$|h|=span\times N_{sample}+1$,   为使采样判决不出现偏差需将判决点右移$|h|$个单位
        \item 关于发送滤波器对SNR的影响,   若其在BPSK下的影响为$\alpha$,   则由于发送信号为累加的关系,   在QPSK下它的影响为$\alpha^2$。具体数值$\alpha$为发送滤波器的函数。由于并无太多时间分析matlab函数$rcosdesign$的具体原理,   其数值$\alpha=0.5$为多次尝试后所得结果,   并经验证其与span的具体数值无关。一个可能有用的信息为$\sum_{i=1}^{|h|}h^2(i)=1$
      \end{enumerate}
\end{document}
