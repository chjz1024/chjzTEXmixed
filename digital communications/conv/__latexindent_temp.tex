\documentclass[12px]{article}
\usepackage{ctex, graphicx, float, listings}
\setkeys{Gin}{width=\textwidth}
\lstset{language=Matlab, %
    %basicstyle=\color{red}, 
    breaklines=true, %
    morekeywords={matlab2tikz}, 
    keywordstyle=\color{blue}, %
    morekeywords=[2]{1},  keywordstyle=[2]{\color{black}}, 
    identifierstyle=\color{black}, %
    stringstyle=\color{mylilas}, 
    commentstyle=\color{mygreen}, %
    showstringspaces=false, %without this there will be a symbol in the places where there is a space
    numbers=left, %
    numberstyle={\tiny \color{black}}, % size of the numbers
    numbersep=9pt,  % this defines how far the numbers are from the text
    emph=[1]{for, end, break}, emphstyle=[1]\color{red},  %some words to emphasise
    %emph=[2]{word1, word2},  emphstyle=[2]{style}, 
}
\begin{document}
  \title{信道编解码}
  \author{PB16061024 陈进泽}
  \date{\today}
  \maketitle
  \newpage

  \section{实验目的}
    \begin{itemize}
      \item 使用MATLAB进行卷积码编/译码器的仿真
      \item 熟练掌握MATLAB软件、语句
      \item 了解卷积码编/译码器的原理、知识
    \end{itemize}
  \section{实验原理}
    \subsection{卷积码编码器}
      \subsubsection{连接表示}
      卷积码由 3 个整数 n ,  k ,  N 描述。 k / n 也表示编码效率(每编码比特所含的信
      息量) ;但 n 与线性分组码中的含义不同, 不再表示分组或码子长度; N 称为约束长度, 
      表示在编码移位寄存器中 k 元组的级数。卷积码不同于分组码的一个重要特征就是编码
      器的记忆性, 即卷积码编码过程中产生的 n 元组, 不仅是当前输入 k 元组的函数, 而且
      还是前面 N  1 个输入 k 元组的函数。实际情况下,  n 和 k 经常取较小的值, 而通过 N
      的变化来控制编码的能力和复杂性
\end{document}