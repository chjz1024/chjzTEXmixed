\documentclass[12pt]{article}
\usepackage{ctex, graphicx, float, amsmath, hyperref}
\usepackage{subfig}
\setkeys{Gin}{width=\textwidth}
\begin{document}
  \title{FSK 传输实验}
  \author{陈进泽 PB16061024}
  \date{\today}
  \maketitle
  \newpage

  \section{实验目的}
    \begin{itemize}
      \item 熟悉 FSK 调制和解调基本工作原理
      \item 掌握 FSK 数据传输过程
      \item 掌握 FSK 正交调制的基本工作原理与实现方法
    \end{itemize}

  \section{实验原理}
    \subsection{调制}
      在二进制频移键控中, 幅度恒定不变的载波信号的频率随着输入码流的变化而切换 (称为高音和低音, 代表二进制的 1 和 0)。通常, FSK 信号的 表达式为:
      \begin{displaymath}
        \begin{array}{cc}
          S_{FSK}=\sqrt{\frac{2E_b}{T_b}}cos(2\pi f_c+2\pi \Delta f)& 0\le t\le T_b \qquad \textrm{(二进制1)}\\
          S_{FSK}=\sqrt{\frac{2E_b}{T_b}}cos(2\pi f_c-2\pi \Delta f)& 0\le t\le T_b \qquad \textrm{(二进制0)}          
        \end{array}
      \end{displaymath}
      其中$2\pi \Delta f$代表信号载波的恒定偏移。

      产生 FSK 信号最简单的方法是根据输入的数据比特是 0 还是 1, 在两个独立的振荡器 中切换。采用这种方法产生的波形在切换的时刻相位是不连续的, 因此这种 FSK 信号称为 不连续 FSK 信号。不连续的 FSK 信号表达式为:
      \begin{displaymath}
        \begin{array}{cc}
          S_{FSK}=\sqrt{\frac{2E_b}{T_b}}cos(2\pi f_Ht+\theta_1)\quad &0\le t\le T_b \qquad \textrm{(二进制1)}\\
          S_{FSK}=\sqrt{\frac{2E_b}{T_b}}cos(2\pi f_Lt+\theta_2)\quad &0\le t\le T_b \qquad \textrm{(二进制0)}          
        \end{array}
      \end{displaymath}
      其实现由图\ref{modulation1}所示:
      \begin{figure}[H]
        \centering
        \includegraphics[width=.6\textwidth]{pict/调制/非连续相位FSK的调制框图.png}
        \caption{非连续相位 FSK 的调制框图}
        \label{modulation1}
      \end{figure}

      由于相位的不连续会造频谱扩展, 这种 FSK 的调制方式在传统的通信设备中采用较 多。随着数字处理技术的不断发展, 越来越多地采用连续相位 FSK 调制技术。

      目前较常用产生 FSK 信号的方法是, 首先产生 FSK 基带信号, 利用基带信号对单一 载波振荡器进行频率调制。因此, FSK 可表示如下:
      \begin{displaymath}
        \begin{array}{rl}
          S_{FSK}(t)&=\sqrt{\frac{2E_b}{T_b}}cos[2\pi f_Ct+\theta(t)]\\
                    &=\sqrt{\frac{2E_b}{T_b}}cos[2\pi f_Ct+2\pi k\int_{-\infty}^tm(t)dt]
        \end{array}
      \end{displaymath}

      应当注意, 尽管调制波形 m(t)在比特转换时不连续, 但相位函数$\theta(t)$是与 m(t) 的积分成比例的, 因而是连续的, 其相应波形如图\ref{modulation2}所示:
      \begin{figure}[H]
        \centering
        \includegraphics{pict/调制/连续相位FSK的调制信号.png}
        \caption{连续相位 FSK 的调制信号}
        \label{modulation2}
      \end{figure}

      由于 FSK 信号的复包络是调制信号 m(t)的非线性函数, 确定一个 FSK 信号的频谱 通常是相当困难的, 经常采用实时平均测量的方法。二进制 FSK 信号的功谱密度由离散频 率分量 $f_c、f_c+n\Delta f、f_c-n\Delta f$ 组成, 其中 n 为整数。相位连续的 FSK 信号的功率谱密度函 数最终按照频率偏移的负四次幂衰减。如果相位不连续, 功率谱密度函数按照频率偏移的 负二次幂衰减。

      FSK 的信号频谱如图\ref{modulation3} 所示。
      \begin{figure}[H]
        \centering
        \includegraphics[width=.6\textwidth]{pict/调制/FSK的信号频谱.png}
        \caption{FSK 的信号频谱}
        \label{modulation3}
      \end{figure}
      FSK 信号的传输带宽 Br, 由 Carson 公式给出:
      \[Br=2\Delta f+2B\]
      其中 B 为数字基带信号的带宽。假设信号带宽限制在主瓣范围, 矩形脉冲信号的带宽 B=R。因此, FSK 的传输带宽变为:
      \[Br=2(\Delta f+R)\]
      如果采用升余弦脉冲滤波器, 传输带宽减为:
      \[Br=2\Delta f+(1+\alpha)R\]

      其中$\alpha$为滤波器的滚降因子。

      在通信原理综合实验系统中, FSK 的调制方案如下:

      FSK 信号:
      \[s(t)=cos(\omega_0t+2\pi f_it)\]

      其中:
      \begin{displaymath}
        f_i=\begin{cases}
          f_1\qquad \textrm{当输入码为} 1\\
          f_2\qquad \textrm{当输入码为} 0
        \end{cases}
      \end{displaymath}

      因而有:
      \begin{displaymath}
        \begin{array}{rl}
          s(t)=&\cos{\omega_0t}\cos{2\pi f_it}-\sin{\omega_0t}\sin{2\pi f_it}\\
              =&\cos{\omega_0t}\cos{\theta(t)}-\sin{\omega_0t}\sin{\theta(t)}
        \end{array}
      \end{displaymath}

      其中:
      \[\theta(t)=2\pi f_ct+2\pi K\int_{-\infty}^tm(t)dt\]

      如果进行量化处理, 采样速率为 $f_s$, 周期为 $T_s$, 有下式成立:
      \begin{displaymath}
        \begin{array}{rl}
          \theta(n)&=\theta(n-1)+2\pi f_cT_s+2\pi Km(n)T_s\\
                  &=\theta(n-1)+2\pi T_s[f_s+Km(n)]\\
                  &=\theta(n-1)+2\pi f_iT_s
        \end{array}
      \end{displaymath}

      按照上述原理, FSK 正交调制器的实现为如图\ref{modulation4}结构:
      \begin{figure}[H]
        \centering
        \includegraphics{pict/调制/FSK正交调制器结构图.png}
        \caption{FSK 正交调制器结构图}
        \label{modulation4}
      \end{figure}

      如果发送 0 码, 则相位累加器在前一码元结束时相位 $\theta(n)$ 基础上, 在每个抽样到达时刻相位累加$2\pi f_1T_s$, 直到该信号码元结束;如发送1码, 则相位累加器在前一码元结束时 的相位 $\theta(n)$ 基础上, 在每个抽样到达时刻相位累加$2\pi f_2T_s$, 直到该码元结束。

      在通信信道 FSK 模式的基带信号中传号采用 $f_H$ 频率, 空号采用 $f_L$ 频率。在 FSK 模式 下, 不采用汉明纠错编译码技术。调制器提供的数据源有:
      \begin{enumerate}
        \item 外部数据输入:可来自同步数据接口、异步数据接口和 m 序列;
        \item 全 1 码:可测试传号时的发送频率(高);
        \item 全 0 码:可测试空号时的发送频率(低);
        \item 0/1 码:0101...交替码型, 用作一般测试;
        \item 特殊码序列:周期为 7 的码序列, 以便于常规示波器进行观察;
        \item m 序列:用于对通道性能进行测试;
      \end{enumerate}

      FSK 调制器基带处理结构如图\ref{modulation5}所示:
      \begin{figure}[H]
        \centering
        \includegraphics[width=.8\textwidth]{pict/调制/FSK调制器基带处理结构示意图.png}
        \caption{FSK调制器基带处理结构示意图}
        \label{modulation5}
      \end{figure}

    \subsection{解调}
      对于 FSK 信号的解调方式很多:相干解调、滤波非相干解调、正交相乘非相干解调。
      \subsubsection{FSK 相干解调}
        FSK 相干解调要求恢复出传号频率$( f_H )$与空号频率$( f_L )$ , 恢复出的载波信号分别 与接收的 FSK 中频信号相乘, 然后分别在一个码元内积分, 将积分之后的结果进行相减,  如果差值大于 0 则当前接收信号判为 1, 否则判为 0。相干 FSK 解调框图如图\ref{demodulation1}所示:
        \begin{figure}[H]
          \centering
          \includegraphics{pict/解调/相干FSK的解调框图.png}
          \caption{相干FSK的解调框图}
          \label{demodulation1}
        \end{figure}

        相干 FSK 解调器是在加性高斯白噪声信道下的最佳接收, 其误码率为:
        \[P_e=Q(\sqrt{\frac{E_b}{N_0}}\]
        相干 FSK 解调在加性高斯白噪声下具有较好的性能, 但在其它信道特性下情况则不完全相同, 例如在无线衰落信道下, 其性能较差, 一般采用非相干解调方案。
      \subsubsection{FSK 滤波非相干解调}
        \begin{figure}[H]
          \centering
          \includegraphics{pict/解调/非相干FSK接收机的方框图.png}
          \caption{非相干FSK接收机的方框图}
          \label{demodulation2}
        \end{figure}
        对于 FSK 的非相干解调一般采用滤波非相干解调, 如图\ref{demodulation2}所示。输入的 FSK 中频 信号分别经过中心频率为 $f_H 、 f_L$ 的带通滤波器, 然后分别经过包络检波, 包络检波的输出在 $t=kT_b$ 时抽样(其中 k 为整数), 并且将这些值进行比较。根据包络检波器输出的大小,  比较器判决数据比特是 1 还是 0。

        使用非相干检测时 FSK 系统的平均误码率为:
        \[P_e=\frac{1}{2}exp(-\frac{E_b}{2N_0})\]

        在高斯白噪声信道环境下 FSK 滤波非相干解调性能较相干 FSK 的性能要差, 但在无 线衰落环境下, FSK 滤波非相干解调却表现出较好的稳健性。

        FSK 滤波非相干解调方法一般采用模拟方法来实现, 该方法不太适合对 FSK 的数字 化解调。对于 FSK 的数字化实现方法一般采用正交相乘方法加以实现。
      \subsubsection{FSK 的正交相乘非相干解调}
        FSK 的正交相乘非相干解调框图如图\ref{demodulation3}所示:
        \begin{figure}[H]
          \centering
          \includegraphics[width=.8\textwidth]{pict/解调/FSK正交相乘非相干解调示意图.png}
          \caption{FSK正交相乘非相干解调示意图}
          \label{demodulation3}
        \end{figure}
        输入的信号为
        \[R(t)=cos(\omega_ct\pm\Delta\omega t)\]
        
        传号频率为: $\omega_c+\Delta\omega$

        空号频率为: $\omega_c-\Delta\omega$

        在上图中, 延时信号为:
        \[R'(t)=cos[(\omega_c\pm\Delta\omega)\cdot(t-\tau)]\]

        其中$\tau$为延时量。

        相乘之后的结果为:
        \begin{displaymath}
          \begin{array}{rl}
            2R(t)\cdot R'(t)&=2cos(\omega_c\pm\Delta\omega)\cdot t*cos[(\omega_c\pm\Delta\omega)\cdot(t-\tau)]\\
                            &=cos[2(\omega_c\pm\Delta\omega)\cdot t-(\omega_c\pm\Delta\omega)\cdot\tau]+cos(\omega_c\pm\Delta\omega)\cdot\tau

          \end{array}
        \end{displaymath}

        在上式中, 第一项经过低通滤波器之后可以滤除。当$\omega_c\cdot\tau=\pi/2$时, 上式可简化为:
        \[2R(t)\cdot R'(t)\approx sin(\pm\Delta\omega)\cdot\tau=\pm sin\Delta\omega\tau\]

        因而经过积分器(低通滤波器)之后, 输出信号大小为:$\pm T_bsin\Delta\omega\tau$, 从而实现了 FSK 的正交相乘非相干解调。

        AB 两点的波形如图\ref{demodulation4}所示:
        \begin{figure}[H]
          \centering
          \includegraphics{pict/解调/差分解调波形.png}
          \caption{差分解调波形}
          \label{demodulation4}
        \end{figure}

        在 FSK 中位定时的恢复见 BPSK 解调方式。

        通信原理实验的 FSK 模式中, 采样速率为 96KHz 的采样速率(每一个比特采 16 个样 点) , FSK 基带信号的载频为 24KHz, 因而在 DSP 处理过程中, 延时取 1 个样值。

        FSK 的解调框图如图\ref{demodulation5}所示:

        注意:FSK 信号首先要和接收端的两个本地正交载波相乘, 然后分别通过低通滤波器 到达 TPJ05 和 TPJ06。图 \ref{demodulation5} 中仅画出低通后的电路。
        \begin{figure}[H]
          \centering
          \includegraphics[width=.7\textwidth]{pict/解调/FSK的解调方框图.png}
          \caption{FSK的解调方框图}
          \label{demodulation5}
        \end{figure}
  \section{实验内容}
    \subsection{FSK 调制}
      \subsubsection{FSK 基带信号观测}
        \begin{figure}[H]
          \centering
          \begin{minipage}[h]{.49\textwidth}
            \centering
            \includegraphics{pict/photo_2018-12-19_18-43-24.jpg}
            \caption{全0码}
          \end{minipage}
          \begin{minipage}[h]{.49\textwidth}
            \centering
            \includegraphics{pict/photo_2018-12-19_18-43-22.jpg}
            \caption{全1码}
          \end{minipage}
        \end{figure}
        如图, 全1码频率为全0码的两倍
      \subsubsection{发端同相支路和正交支路信号时域波形观测}
        \begin{figure}[H]
          \centering
          \begin{minipage}[h]{.49\textwidth}            
            \centering
            \includegraphics{pict/photo_2018-12-19_18-43-08.jpg}\hfill
            \includegraphics{pict/photo_2018-12-19_18-43-02.jpg}
            \caption{全0码}
          \end{minipage}
          \begin{minipage}[h]{.49\textwidth}
            \centering
            \includegraphics{pict/photo_2018-12-19_18-43-06.jpg}\hfill
            \includegraphics{pict/photo_2018-12-19_18-43-04.jpg}
            \caption{全1码}
          \end{minipage}
          \includegraphics[width=.49\textwidth]{pict/photo_2018-12-19_18-43-29.jpg}
          \includegraphics[width=.49\textwidth]{pict/photo_2018-12-19_18-42-59.jpg}
          \caption{m序列}
        \end{figure}
        如图, 由其时域波形及李萨如图形, 可见对于所有序列, TPi03与TPi04均正交

        当产生两路正交信号进行调制时可有效避免01切换时的相位不连续现象, FSK功率谱旁瓣较弱, 占用带宽减少
      \subsubsection{连续相位 FSK 调制基带信号观测}
        \begin{figure}[H]
          \centering
          \includegraphics{pict/photo_2018-12-19_18-43-27.jpg}
          \caption{码元切换点}
        \end{figure}
        如图, 在连续相位调制时码元切换点处相位连续。若是非连续相位FSK调制的话在切换点处会有180度的相差
    \subsection{FSK 解调}
      \subsubsection{解调基带 FSK 信号观测}
        \begin{figure}[H]
          \centering
          \begin{minipage}[h]{.49\textwidth}            
            \centering
            \includegraphics{pict/photo_2018-12-19_18-43-22.jpg}\hfill
            \includegraphics{pict/photo_2018-12-19_18-43-13.jpg}
            \caption{全1码}
          \end{minipage}
          \begin{minipage}[h]{.49\textwidth}
            \centering
            \includegraphics{pict/photo_2018-12-19_18-43-20.jpg}\hfill
            \includegraphics{pict/photo_2018-12-19_18-43-11.jpg}
            \caption{01码}
          \end{minipage}
        \end{figure}
        如图, 波形因为有噪声的干扰有所变化, 但大致上仍为正交且有特定频率的两路正弦波
      \subsubsection{接收位同步信号相位抖动观测}
        \begin{figure}[H]
          \centering
          \begin{minipage}[h]{.49\textwidth}
            \centering
            \includegraphics{pict/photo_2018-12-19_18-42-53.jpg}
            \caption{01码}
          \end{minipage}
          \begin{minipage}[h]{.49\textwidth}
            \centering
            \includegraphics{pict/photo_2018-12-19_18-42-51.jpg}
            \caption{全0码}
          \end{minipage}
        \end{figure}
        实际测试过程中全0码或全1码获得的尖峰脉冲在不断变化, 其理由为不像01码在输入跳变时有同步信息的出现, 全0或全1码并不存在可以调整接收时钟的信号, 因此无法正确同步
      \subsubsection{解调器位定时恢复与最佳抽样判决点波形观测}
        \begin{figure}[H]
          \centering
          \includegraphics{pict/photo_2018-12-19_18-42-33.jpg}
        \end{figure}
        抽样判决点基本上在波形的正中间
      \subsubsection{位定时锁定和位定时调整观测}
        \begin{figure}[H]
          \centering
          \begin{minipage}[h]{.49\textwidth}
            \centering
            \includegraphics{pict/photo_2018-12-19_18-42-27.jpg}
            \caption{m序列}
          \end{minipage}
          \begin{minipage}[h]{.49\textwidth}
            \centering
            \includegraphics{pict/photo_2018-12-19_18-42-18.jpg}
            \caption{全0序列}
          \end{minipage}
        \end{figure}
        如图, 仅当输入信号为m序列时恢复时钟才会稳定, 且当断开JL02后无论输入信号为什么均不能使得接收时钟稳定, 理由为全0或全1序列与无接受信号一样, 无法提取出同步信息, 便没有可能获得反馈来调整相位
  \section{结论}
    FSK正交调制产生两路信号用于调制, 相较与一般FSK调制, 其具有相位连续特性, 可有效地减少带宽
\end{document}
  