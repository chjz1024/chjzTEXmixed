\documentclass[utf8]{article}
\usepackage{ctex}
\usepackage{graphicx}
\usepackage{float}
\usepackage{cite}
\usepackage{amsmath}
\usepackage{hyperref}

\setkeys{Gin}{width=\textwidth}
\begin{document}
  \title{Pruning in Deep Compression}
  \author{Jinze Chen}
  \date{\today}
  \maketitle
  
  \section{Introduction}
    % People design pruned neural networks mainly for 2 reasons: one for the deployment in resource constrained devices,  \textit{e.g.},  mobile phones or embedded gadgets and the other to reduce the effect subjected to over-parameterization,  thus this approach may even increase the inference accuracy in certain networks.
    对网络进行剪枝主要有两个目的:一是降低网络的内存与计算资源消耗以便于将其部署在资源有限设备上, 二是减轻过拟合现象, 因此剪枝可能提高某些网络精度。
    
    % For the latter one,  this process resembles the biological phenomena in mammalian brain,  where the number of neuron synapses has reached the peak in early childhood,  followed by gradual pruning during its development. However,  we don't concern much about increasing the accuracy in this article. Having a new sparse network that functions not so strayed from the original one is our ultimate goal.
    对于后一种情况, 其整个流程类似于哺乳动物脑的发育过程, 在这过程中脑神经元的突触数量在儿童期间到达顶峰, 而后逐渐减少。然而本文并不对该内容进行阐述, 我们的最终目标是以更少的资源消耗尽可能地逼近目标网络。本文罗列了一些其他人在这方面做的努力, 将其集合起来, 并总结出网络剪枝必要的流程以及其注意事项。

  \section{Related work}
    % The main problem is the principle to find the appropriate mask in the whole network. Different choices may make different trade-offs between computational resources and accuracy,  \textit{e.g.},  weight-wise pruning may achieve better compression rate while layer-wise pruning is much easier to be performed. There are also some attempts to reconstruct a similar network but with fewer parameters or layers\cite{Li2018DeepRebirthAD},  which we'll examine later in \ref{DeepRebirth}.
    在网络剪枝中最关键的问题是如何判别网络中不重要的成分并将其舍弃, 其中对计算资源以及网络精度的权衡也是很重要的一部分, 像是以权重为单元的剪枝通常精度较高, 但计算较慢, 相比之下更高级别的如层级别的剪枝则正好相反。还有一些通过改变原网络结构来降低存储量的做法, 具体实现将在\ref{DeepRebirth}里描述。

    % In summary, less significant compositions may be masked safely without affecting the network performance. Based on this principle,  Han et al. \cite{Han2015LearningBW} proposed an iterative pruning method to remove the redundancy in deep models. Their main insight is that small-weight connectivity below a threshold should be discarded. In practice,  this can be aided by applying $l_1$ or $l_2$ regularization to push connectivity values becoming smaller. The major weakness of this strategy is the loss of universality and flexibility,  thus seems to be less practical in the real applications.
    总结起来越不重要的元素被舍弃时对网络的影响越小, 基于该原则韩松在\cite{Han2015LearningBW}中提出了一种迭代剪枝的方法来压缩网络, 在这过程中绝对值小于一阈值的权重全被舍弃。实际应用时可以通过$l_1$与$l_2$正则化来更方便地实现。这种方法的主要缺陷是缺少通用性与灵活性。

    % In order to avoid these weaknesses,  some attention has been focused on the group-wise sparsity. Lebedev and Lempitsky \cite{Lebedev2016FastCU} explored group-sparse convolution by introducing the group-sparsity regularization to the loss function,  then some entire groups of weights would shrink to zeros,  thus can be removed. Similarly,  Wen et al. \cite{Wen2016LearningSS} proposed the Structured Sparsity Learning (SSL) method to regularize filter,  channel,  filter shape and depth structures. In spite of their success,  the original network structure has been destroyed. As a result,  some dedicated libraries are needed for an efficient inference speed-up.
    为了克服上述缺点, 也有些将注意力转移到了``组''层面的稀疏性。Lebedev与Lempitsky在他们的论文\cite{Lebedev2016FastCU}中通过对损失函数的正则化来去除一些权重组。与此类似, Wei Wen在其论文\cite{Wen2016LearningSS}中提出了Structured Sparsity Learning (SSL)方法来正规化filter, channel, filter shape以及depth structure。但是在该过程中原有网络结构已被损坏, 其结果是需要特殊的函数库来加速网络。

    % Some filter level pruning strategies have been explored too. The core is to evaluate neuron importance,  which has been widely studied in the community \cite{Hu2016NetworkTA, Li2016PruningFF, Molchanov2016PruningCN, Selvaraju2017GradCAMVE, Zhou2016LearningDF}. A simplest possible method is based on the magnitude of weights. Li et al. \cite{Li2016PruningFF} measured the importance of each filter by calculating its absolute weight sum. Another practical criterion is to measure the sparsity of activations after the ReLU function. Hu et al. \cite{Hu2016NetworkTA} believed that if most outputs of some neurons are zero,  these activations should be expected to be redundant. They compute the Average Percentage of Zeros (APoZ) of each filter as its importance score. These two criteria are simple and straightforward,  but not directly related to the final loss. Inspired by this observation,  Molchanov et al. \cite{Molchanov2016PruningCN} adopted Taylor expansion to approximate the influence to loss function induced by removing each filter.
    也有一些人以filter层面的剪枝为基础, 其研究关键点在与衡量神经元重要性, 这已经在\cite{Hu2016NetworkTA, Li2016PruningFF, Molchanov2016PruningCN, Selvaraju2017GradCAMVE, Zhou2016LearningDF}中被广泛研究了。最简单的方法是单纯根据其绝对值来衡量, 其中Hao Li在其论文\cite{Li2016PruningFF}中以绝对和为衡量标准。还有的比较实用的方法是通过根据通过ReLU函数后的激活率来判断, 如Hengyuan Hu在\cite{Hu2016NetworkTA}中的研究成果。这两种方法直接且简洁, 但它们并未将操作与结果直接联系起来。受此启发, Pavlo Molchanov在其论文\cite{Molchanov2016PruningCN}中使用泰勒级数展开来分析移除filter对最终损失的影响。

  \section{Different methodologies}
    \subsection{Neuron Pruning for Compressing Deep Networks using Maxout Architectures\cite{Rueda2017NeuronPF}}
    %This paper presents an approach for reducing the size of deep neural networks by pruning entire neurons. It also can be combined with subsequent weight pruning.
本论文提出了一种剪除整个神经元的方案, 在该操作后可以进一步地进行权重剪枝。

该方案主要是是哟你个了maxout单元以及其model selection能力来剪掉整个神经元, 从而压缩整个网络。该过程假设maxout单元中存在冗余, 具体流程如下图:
\begin{figure}[H]
  \centering
  \includegraphics{subsection/maxout.png}
\end{figure}
%First,  a CNN with a maxout layer is trained. This maxout layer performs a max function among k adjacent neurons,  reducing the amount of weights connecting with the next layer by a factor of k. So,  placing this maxout layer after the one with the highest number of weights would be advisable. Second,  by counting the number of times neurons become the maximal value in each maxout unit when computing a forward pass over the training dataset,  the least active neurons of each maxout unit are removed from the network. Their effects are negligible with respect to other neurons. Third,  the remaining neurons of the CNN are re-trained. After re-training,  the process is repeated;
首先训练一个包含maxout层的CNN。该maxout层在k个相继的神经元间取最大值, 连接数得以减少k倍, 因此将该maxout层置于最多权重的曾会得到较好效果。再来, 通过记录某一神经元在maxout单元中被激活的次数, 激活次数较小的神经元被移除, 此时该神经元对整个网络的影响可以忽略。最后该CNN被重新训练, 训练完成后重复上述过程。

%In weight pruning,  the weights get pruned by thresholding them. This is one simple approach,  and it can be replaced by other criterions like relevance measure using Hessian matrix.
权重剪枝中绝对值较低的权重被忽略。该过程可以以其他准则替代。

%This approach reduced the network size by up to 74\% in LeNet-5 and 61\% in VGG16 without affecting the network's performance,  only applying the neuron pruning.
仅进行神经元剪枝的情况下, LeNet-5的网络体积最多降低了74\%, VGG16降低了61\%而网络表现不降低。

%\textbf{The main insight in this paper is considering redundancies in maxout architecture. Because this can be combined with other weight pruning methods,  it's not bad to combine this approach in certain networks.}
\textbf{该论文主要的创新点是考虑maxout结构的冗余并对其处理。因为该过程可以与其他权重剪枝方法结合, 对于具有某些结构的网络可以考虑使用该方案。}
    \subsection{ThiNet: A Filter Level Pruning Method for Deep Neural Network Compression\cite{Luo2017ThiNetAF}}
    % This paper proposed a filter level pruning framework,  called ThiNet,  to simultaneously accelerate and compress CNN models in both training and inference stages.
该论文提出了filter层面的网络剪枝方案, 被称为ThiNet。使用这种方式可以在训练与推测阶段加速与压缩原网络。

% The idea is to establish filter pruning as an optimization problem,  \textit{i.e.},  to find a subset of channels in layer (i+1)'s (not i's) input to approximate the output in layer i+1. Instead of considering which filter to discard,  the author finds the most important filters to be preserved.
本论文主要思路是将filter剪枝视为一优化问题, 即在i+1层的channel中找到一子集来近似模拟i+1曾的输出。相对于考虑该舍弃哪些filter, 作者考虑的是保留哪些重要filter。

% Given a pre-trained model,  it would be pruned layer by layer with a predefined compression rate. The process is shown below:
该方法对一已训练好的模型根据设定好的压缩率逐层进行进行剪枝, 整个过程示意如下:
\begin{figure}[H]
  \centering
  \includegraphics{subsection/ThiNet.png}
\end{figure}

% \begin{enumerate}
  % \item A training set is randomly sampled.
  % \item Using a greedy algorithm for channel selection,  \textit{i.e.},  to minimize \[\arg\min_T \sum_{i=1}^M \left(\hat{y}_i-\sum_{j\in S}\hat{x}_{i, j} \right)^2\quad s.t.\quad |S|=C\times r,  S\in \{1, 2, \ldots, C\} \]
  % \item Minimize the reconstruction error by weighing the channels,  which can be defined as $\hat{w}=\arg\min_w\sum_{i=1}^M(\hat{y}_i-w^T\hat{x}_i^*)^2$
  % \item Iterate to step 1 to prune the next layer.
% \end{enumerate}
\begin{enumerate}
  \item 随机选出一样本集。
  \item 使用贪婪算法来选择channel, 以最小化 \[\arg\min_T \sum_{i=1}^M \left(\hat{y}_i-\sum_{j\in S}\hat{x}_{i, j} \right)^2\quad s.t.\quad |S|=C\times r,  S\in \{1, 2, \ldots, C\} \]
  \item 对各channel加权来最小化误差, 定义如下 $\hat{w}=\arg\min_w\sum_{i=1}^M(\hat{y}_i-w^T\hat{x}_i^*)^2$
  \item 迭代至步骤一来对下一层剪枝。
\end{enumerate}

作者使用这种方式在VGG-16上以0.52\%的top-5准确度下降换来了$3.31\times$的FLOPs下降以及$16.63\times$的压缩率。

% \textbf{The main insight is that the author establishes a well-defined optimization problem,  which shows that whether a filter can be pruned depends on the outputs of its next layer,  not its own layer. Also because this approach prunes the network in filter level,  the computational cost is relatively low,  compared with weight pruning.}
\textbf{本文主要创新点是作者将剪枝问题视为一优化问题, 其中各filter的重要性以其到下一层的输出来衡量而不是本层。并且由于剪枝是在filter层面进行的, 计算复杂度与权重剪枝相比相对较低。}
    \subsection{Sparsifying Neural Network Connections for Face Recognition\cite{Sun2016SparsifyingNN}}
    %This paper focused on creating a new network for face recognition,  while also sparsified the whole network after the training has finished. What's meaningful to our work is the principle he used to prune weights.
该论文主要内容是针对人脸识别应用设计一个新的网络, 并在网络训练完成后对该网络进行剪枝操作来。对我们来说有用的只有其剪枝内容。

%The author tends to keep connections (and the corresponding weights) where neurons connected have high correlations and drop connections between weakly correlated neurons. For fully and locally-connected layers,  where weights are not shared,  given a neuron $a_i$ in the current layer and its K connected neurons $b_{i1} ,  b_{i2} , \ldots , b_{iK}$ in the previous layer,  the correlation coefficient between $a_i$ to each of $b_{ik}$ for $k = 1,  2,  \ldots ,  K$ is
%\begin{displaymath}
%  r_{ik}=\frac{E[a_i-\mu_{a_i}][b_{ik}-\mu_{b_{ik}}]}{\sigma_{a_i}\sigma_{b_{ik}}}
%\end{displaymath}
作者计算神经元间相关性, 并舍弃相关性较低神经元间的链接。对与全连接层等权重不共享的层, 给定一神经元$a_i$与其前一层中K个相连的神经元$b_{i1} ,  b_{i2} , \ldots , b_{iK}$, $a_i$与$b_{ik}, k = 1, 2, \ldots, K$间的相关为:
\begin{displaymath}
  r_{ik}=\frac{E[a_i-\mu_{a_i}][b_{ik}-\mu_{b_{ik}}]}{\sigma_{a_i}\sigma_{b_{ik}}}
\end{displaymath}
%where $\mu_{a_i}, \mu_{b_{ik}}, \sigma_{a_i}, \sigma_{b_{ik}}$denote the mean and standard deviation of $a_i$ and $b_{ik}$,  respectively,  which are evaluated on a separated training set. Since both positively and negatively neurons are important,  the author used similar but slightly different methods to deal with them. For convolutional layers,  the author calculated the mean magnitude,  which we don't present here.
其中$\mu_{ai}, \mu_{b_{ik}}, \sigma_{a_i}, \sigma_{b_{ik}}$分别代表$a_i$与$b_{ik}$的均值与标准差, 其值在另一训练集中获得。对于正和负的相关值, 作者使用了不同的方式处理。对于卷积层作者使用另一公式来计算, 在这里我们不展现。

%\textbf{The main insight is to measure weight importance based on the previous layer. The author also experimented on directly training the sparse ConvNet from scratch,  which failed to find good solutions for face recognition. Although we mainly focus on pruning an existing network,  this could be some hint to the training stage.}
\textbf{本论文主要的创新点是根据层间关系权衡重要性。除此之外, 作者也尝试了从头开始训练一稀疏网络, 结果在人脸识别上效果并不好。尽管我们的关注点是对已知网络进行剪枝, 该结论仍具有很强的指导意义。}
    \subsection{DeepRebirth: Accelerating Deep Neural Network Execution on Mobile Devices\cite{Li2018DeepRebirthAD}\label{DeepRebirth}}
    % This paper finds another solution for speeding up model inference and reducing model size,  which is called DeepRebirth. Strictly speaking this method shouldn't be called pruning,  but it's useful for deployment in mobile phones and has been tested on some mobile devices,  so we list it here.
本论文找到了一种提高模型预测速度与减小模型体积的方法, 被称为DeepRebirth。严格来说这一方法不应被称为剪枝, 但是它对于在移动端的部署是十分有用的, 并且作者在实际的手机上测试了这一结果, 因此我们把这种方法列了出来。

% First the author found that non-tensor layers consume too much time in model execution,  which is shown below:
首先作者发现非张量层在运行中消耗很多时间, 如下图:
\begin{figure}[H]
  \centering
  \includegraphics{subsection/DeepRebirth.png}
\end{figure}
% So the author thought of combining those non-tensor layers with its adjacent tensor layers. This paper presents two ways to combine those layers,  called \textbf{Streamline Slimming} and \textbf{Branch Slimming} respectively.
因此作者想要通过将非张量层与张量层合并为一层来提高运行效率。这篇文章提出了两种合并方法, 分别为\textbf{Streamline Slimming}和\textbf{Branch Slimming}。
\begin{figure}[H]
  \centering
  \begin{minipage}[h]{.49\textwidth}
    \centering
    \includegraphics{subsection/overall.png}
  \end{minipage}
  \begin{minipage}[h]{.49\textwidth}
    \centering
    \includegraphics{subsection/Streamline.png}
  \end{minipage}
\end{figure}
\begin{figure}
  \centering
  \includegraphics{subsection/Branch.png}
\end{figure}

% The weights and biases are retrained to attain similar performance with original network,  which is given by
权重与偏移经过重训练来获得与原网络相似的效果, 如下式:
\begin{displaymath}
  (\tilde{W}^*, \tilde{B}^*)=\underset{W, B}{\arg\min}\sum_i ||Y_{CNN}^i-\tilde{f}(W, B;X^i)||_F^2
\end{displaymath}
where $W$ and $B$ represent weight and bias matrix respectively.
其中$W$和$B$分别代表权重与偏移矩阵。

% \textbf{The main insight in this article is to evaluate execution time of non-tensor layers and solve this problem by merging certain layers,  and it has achieved great success in speeding up the whole network. However,  this is a completely engineering practice. It could be added to any network after the training has finished.}
\textbf{本文主要的创新点是分析得到了非张量层的运行时间很长并通过合并特定层解决了该问题, 然而该方式完全为工程方向, 并不具有太多理论指导意义。好处是这一做法可以用于任何训练后的网络部署中。}

  \section{Summary}
    % These articles all present some ways to accelerate or compress neural networks,  or both. In general,  we may derive the process for pruning an existing network as follows:
    % \begin{enumerate}
      % \item Get a trained network.
      % \item Prune some components based on its importance. It's better to evaluate that based on its relation with output rather than on one layer exclusively.
      % \item Retrain the whole network to minimize error induced by pruning. After that iterate to step 2.
      % \item [*4] Reconstruct a new network for less model latency. 
    % \end{enumerate}

    这些论文都提出了一些加速或压缩神经网络的方案。大体上来说, 我们可以得出对一已知网络剪枝流程如下:
    \begin{enumerate}
      \item 获取一训练好的神经网络。
      \item 根据其组成元素重要性进行剪枝, 其中通过层间重要性来判断通常能获得更好的效果。
      \item 对该网络进行重训练来最小化剪枝误差, 然后回到步骤2.
      \item [*4] 重构一网络来降低模型延迟。
    \end{enumerate}

    % In order for step 4 to be combined,  there needs some extra efforts,  or we could only focus on model speed or model size. Anyway,  that's the main framework for network pruning. Also we could combine other methods like quantization and entropy coding to reduce the model size further.
    为了进行步骤4, 我们需要在其他一些方面入手, 或者我们只关注模型速度或模型大小。无论如何, 上述过程应该就是网络剪枝的大体流程, 而在整个过程中我们还可以采用如量化与熵编码等错吃来进一步压缩模型。
    
  \bibliographystyle{plain}
  \bibliography{ref}
\end{document}