预测编码是视频编码的核心技术之一。对于视频信号来说,   一幅图像邻近像素间有较强空间相关性,   相邻图像间有很强时间相关性。因此,   先进的视频编码常采用帧内预测和帧间预测的方式,   去除空域和时域的相关性,   之后编码器对预测后的残差进行变换,   量化,   熵编码,   可以大幅提高编码效率。

原理用信息论可以解释如下:对于独立信源X与Y, 可以证明$H(X+Y)\geq max(H(X),H(Y))$。对于帧内预测假设图像由某一纹理X与随机残差Y构成, 而针对于帧间预测假设图像由其参考图像X与随机残差Y构成。对整体信源进行压缩信息量为$H(X+Y)$, 若选择一合适纹理进行编码则原信源信息为$\frac{1}{n}H(X)+H(Y)$, 由上述不等式当$n\to\infty$时必有去相关的编码模式压缩效率更高。

预测流程如下:
\begin{enumerate}
    \item 以同一帧图像内的临近像素作为参考,   计算预测值Xp
    \item 原始值X和预测值Xp的差值d,   被传递到解码端
    \item 解码端接收到差值d,   将其与预测值Xp相加,   就得到了``原始值''X',   X'=Xp+d
\end{enumerate}

整个过程可以可以简化为下图\ref{PC:1}:
\begin{figure}[H]
    \centering
    \includegraphics[width=.6\textwidth]{pict/PC/Intra/1.png}
    \caption{流程图}
    \label{PC:1}
\end{figure}
该方法同时适用于帧内预测和帧间预测。
\subsection{帧内预测}
\subsubsection{简述}
视频序列中的每帧图像,   局部会有描述同一物体的情况,   而描述同一物体的相邻像素间就会有相关性,   并不是绝对独立的。
    
这时,   用周围已编码部分表示(预测)当前所需编码的部分,   这种在空间域上进行的预测编码算法,   可以除去相邻块之间的空间冗余度,   取得更为有效的压缩。

对于存在前后相关性的信息,   预测编码是一种非常简便且有效的方法。此时预测编码输出的不再是原始的信号值,   而是信号的预测值与实际值的差。预测编码如此设计的出发点在于,   由于前后存在相关性,   相邻信号存在大量相同或相近的现象,   通过计算其差值,   可以减少大量保存与传输原始信息的数据体积。

如图\ref{PC:2}所示
\begin{figure}[H]
    \centering
    \includegraphics[width=.6\textwidth]{pict/PC/Intra/2.png}
    \caption{示例图片\cite{万帅2014新一代高效视频编码}}
    \label{PC:2}
\end{figure}
左图待预测区域内容较为平坦,   可用参考像素取平均的DC模式,   而右图纹理呈水平状排列,   可以采用水平的预测模式。
\subsubsection{帧内预测技术(基于H.264标准的亮度信号预测)}
%\subsubsubsection{基于H.264标准的亮度信号预测}

\paragraph{4x4亮度块,   九种预测模式}

在H.264中用九种预测模式,   分别是:垂直预测,   水平预测,   DC预测,   以及五种不同倾斜方向的预测模式。图\ref{PC:3}中4×4亮度块的上方和左方像素A--M为已编码和重构像素,   用作编解码器中的预测参考像素。a--p为待预测像素,   利用A--M值和9种模式实现。
\begin{figure}[H]
    \centering
    \includegraphics{pict/PC/Intra/33.png}
     \caption{$4\times4$亮度块,   九种预测模式\cite{万帅2003新一代视频压缩标准}}
      \label{PC:3}
\end{figure}
\begin{figure}[H]
   \centering
   \includegraphics[width=.5\textwidth]{pict/PC/Intra/4.png}
   \caption{samples}
   \label{PC:4}
\end{figure}
\paragraph{16x16帧内预测模式}
如图\ref{PC:5},   帧内预测有垂直,   水平,   直流DC和planar四种模式。
\begin{figure}[H]
   \centering
   \includegraphics{pict/PC/Intra/5.png}
   \caption{16x16帧内预测模式}
    \label{PC:5}
\end{figure}

\subsubsection{选择预测模式的标准:XP和X’差值越小越好}
为了选择出合适的帧内预测模式,   H.264/AVC采用了拉格朗日率失真优化(RDO)进行模式选择。它为每一种 模式计算拉格朗日代价:
\begin{equation}
J=D+\lambda\times R
\end{equation}
其中,   D表示当前预测模式下的失真,   R表示编码当前预测模式下所有信息(如变换系数,   模式信息,   宏块划分方式等)所需比特数,   为拉格朗日因子。

需要说明的是,   最优的预测模式不一定残差最小,   而是残差信号经其他编码模块(变换,   量化,   熵编码等)后最终编码性能最优。
\subsubsection{H.265/HEVC的精进之处}
H.265是新的编码协议,   也即是H.264的升级版。H.265标准保留H.264原来的某些技术,   同时对一些相关的技术加以改进。新技术使用先进的技术用以改善码流、编码质量、延时和算法复杂度之间的关系,   达到最优化设置。

\paragraph{H.265帧内预测概述}

在H.265中,   $4\times 4$块预测模式从9种增加至35种,   35种预测模式是在PU的基础上定义的,   而具体帧内预测过程的实现则是以TU为单位的。

H.265/HEVC帧内预测可分为以下3个步骤:
\begin{itemize}
\item 判断当前TU相邻参考像素是否可用并做相应的处理
\item 对参考像素进行滤波
\item 根据滤波后的参考像素计算当前TU的预测像素值
\end{itemize}

\paragraph{相邻参考像素的选取}

如图\ref{PC:6},   当前的TU大小为NxN,   其参考像素按区域可分为5部分:左下(A)、左侧(B)、左上(C)、上方(D)和右上(E), 一共4N+1个点。若当前TU位于图像边界,   则相邻参考像素可能会不存在或不可用。另外,   在某些情形下A或E所在的块可能尚未进行编码,   此时这些参考像素也是不可用的。
\begin{figure}[H]
    \centering
    \includegraphics[width=.6\textwidth]{pict/PC/Intra/6.png}
    \caption{参考像素的选取}
    \label{PC:6}
\end{figure}
当参考像素不存在或不可用时,   H.265/HEVC标准会使用最邻近的像素进行填充。例如,   若区域A的参考像素不存在,   则区域A所有参考像素都用区域B最下方的像素进行填充;若区域E的参考像素不存在,   则区域E所有参考像素都用D最右侧的像素进行填充。需要说明的是,   若所有参考像素都不可用,   则参考像素都用固定值填充,   该固定值大小为
\begin{equation}
R=1\leq (bitdepth-1)
\end{equation}
\paragraph{参考像素的滤波}

由于帧内预测过程中,   边缘会发生突变,   这时就需要滤波。滤波的目的在于提升帧内预测的像素块的视觉效果,   减小边缘可能产生的突变感。是否对参考像素进行滤波取决于帧内预测模式和预测像素块的大小。 
滤波一般是通过一个三抽头滤波器实现,   三抽头滤波器系数为[0.25, 0.5, 0.25]。

对于DC模式以及4*4大小的TU都不需要对参考像素滤波处理,   但是其他部分模式由于特殊需要,   要进行滤波处理,   这里不赘述。
\paragraph{预测像素的计算}

与h.264/avc相比,   h.265/hevc增加使用了左下方块的边界像素作为当前块的参考。这是由于h.264/avc以固定大小的宏块为单元进行编码,   在对当前块进行帧内预测时,   其左下方块很有可能尚未进行编码,   无法用于参考;而h.265/hevc四叉树形的编码结构使得这一区域成为可用像素。此外,   这一区域像素的使用也提供了更多可能的预测方向,   在某些情形下(如倾斜向上方向的纹理等)能够大幅度提高预测精度。

\subsubsection{HEVC帧内预测模式}
H.265/HEVC亮度分量帧内预测支持5种大小的PU: 4x4,  8x8,  16x16,  32x32,  64x64
每一种大小的PU都有35种预测模式:

\paragraph{Planar模式}

Planar模式是由H.264/AVC中的Plane模式发展而来的,   它适用于图像值缓慢变化的区域。Planar模式使用水平和垂直方向的两个线性滤波器,   并将二者的平均值作为当前块像素的预测值。
\begin{figure}[H]
    \centering
    \includegraphics{pict/PC/Intra/7.png}
     \caption{Planar模式}
      \label{PC:7}
\end{figure}
\begin{figure}[H]
    \centering
    \includegraphics{pict/PC/Intra/8.png}
     \caption{Planar模式}
      \label{PC:8}
\end{figure}
算法如下:
\begin{align}
P_{x, y}^H=(N-x)R_{0, y}+x\times R_{n+1, 0}\\
P_{x, y}^V=(N-y)R_{0, x}+y\times R_{0, N+1}\\
P_{x, y}=(P_{x, y}^H+P_{x, y}^V+N)\div 2N    
\end{align}
\paragraph{DC模式}

DC模式适用于大面积平坦区域,   其做法与H.264/AVC基本相同。当前块预测值可由其左侧和上方(注意不包含左上角、左上方和右上方)参考像素的平均值得到。
\begin{figure}[H]
    \centering
    \includegraphics{pict/PC/Intra/9.png}
    \caption{DC模式}
    \label{PC:9}
\end{figure}
算法如下:
\begin{equation}
DCvalue=\sum_{x=1}^NR_{x, 0}+\sum_{y=1}^NR{0, y}\div2N
\end{equation}
左上角像素:
\begin{equation}
P_{1, 1}=(R_{1, 0}+R_{0, 1})\div 2
\end{equation}
第一行像素:
\begin{equation}
{P_{x, 1}=(R_{x, 0}+3\times DcValue)\div 4}
\end{equation}
第一列像素:
\begin{equation}
P_{1, y}=(R_{0, y}+3\times DcValue)\div4
\end{equation}
其他像素:
\begin{equation}
P_{x, y}= DcValue
\end{equation}
\paragraph{三十三种角度模式}

H.264/AVC使用了8中不同的预测方向(4x4大小),   H.265/HEVC则进一步细化了这些预测方向,   规定了33种角度预测模式,   以更好地适应视频内容种不同方向的纹理。
下图\ref{PC:10}给出了33种角度模式的具体方向,   其中V0(模式26)和H0(模式10)分别表示为垂直和水平方向,   其余模式的预测方向都可以看成再垂直或水平方向上做了一个偏移,   该偏移角的大小可由模式下方的数字计算得出。
\begin{figure}[H]
    \centering
    \includegraphics{pict/PC/Intra/10.png}
     \caption{$\theta=arctan(\frac{x}{32})$}
      \label{PC:10}
     
\end{figure}

$\theta$ 为正表示预测方向向左偏移,   $\theta$ 为负表示预测方向向右偏移;对于水平类模式,   $\theta$ 为正表示预测方向向上偏移,   $\theta$ 为负表示预测方向向下偏移
算法如下:

33种角度模式可分为水平类模式(2--17)和垂直类模式(18--34)

下面以垂直类模式为例给出预测像素计算值:
首先,   对于部分垂直类模式,   即offset(M)<0的模式,   既需要用到左侧像素又需要用到上层像素,   带来了编码的复杂性。有没有一种方法能简化这种复杂度呢,   答案是有的。
这里,   我们就需要将左侧像素映射到上层,   如图\ref{PC:11}所示:

\begin{figure}[H]
    \centering
    \includegraphics[width=.8\textwidth]{pict/PC/Intra/11.png}
    \caption{映射示意图}
    \label{PC:11}
\end{figure}
建立一维数组Ref,   

Offset(M)<0时,   

$$ Ref(x)=\left\{
\begin{aligned}
 R_{0, x},  x\geq0\\
 R_{y(x), 0}, x<0
\end{aligned}
\right.
$$


其中
\begin{equation}
y(x)=round(\frac{32\times x}{offset(M)})
\end{equation}
Offset(M)>0时,   
\begin{equation}
Ref(x)=R_{0, x}    , x=0, 1, 2, 3, ………2N
\end{equation}
接下来建立当前像素对应参考像素在ref里的位置:\\
\begin{equation}
Pos=floor((x\times offset(M))\div32)
\end{equation} \\
加权因子:$W=(x\times offset(M)\&31$,   其中\&表示按位与运算,   可以理解为提取后四位。
最后计算当前像素值:
\begin{equation}
P_{x, y}=((32-w)\times ref[y+pos]+w\times ref[y+pos+1])\div32
\end{equation}
需要注意的是,   对于模式26(垂直模式),   预测像素值为:
\begin{equation}
\left\{
\begin{aligned}
  P_{x, y}&=R_{0, y}\\
 P_{x, y}&=R_{x, 0}\div2+R_{0, 0}
\end{aligned}
\right.
\end{equation}
35种预测模式实现效果图如下\ref{PC:12}:
\begin{figure}[H]
    \centering
    \includegraphics[width=.8\textwidth]{pict/PC/Intra/texture.png}
    \caption{效果图}
    \label{PC:12}
\end{figure}