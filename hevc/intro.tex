\documentclass[utf8, a4paper, 12pt]{article}
\usepackage{ctex}
\usepackage{graphicx} % figure
\usepackage{float}
\usepackage{cite}
\usepackage{amsmath}
\usepackage{verbatim}
\usepackage{chngcntr} % counterwithin
\usepackage{mathrsfs} % mathcal, mathscr
\usepackage{subfig} % subfigure
\usepackage{inputenc} % accent
\usepackage{fancyhdr} % fancey heading
\usepackage[headings]{fullpage}
\usepackage{multirow}
\usepackage{hyperref}


\pagestyle{fancy}

\counterwithin{figure}{section}
\counterwithin{table}{section}
\counterwithin{equation}{section}
%\numberwithin{equation}{section}

\setcounter{tocdepth}{4}
\setcounter{secnumdepth}{4}

\setkeys{Gin}{width=\textwidth}
\begin{document}
  CABAC: hard to parallelize and vectorize

  bit error spreads

  p197,  QP prediction

  编码效率提升50

  dct能量集中性

  bib:iso, 新一代..., High Efficiency Video Coding (HEVC): Algorithms and Architectures, AI: https://zhuanlan.zhihu.com/p/40034222, chapter 6, dct: https://blog.csdn.net/u012596983/article/details/39925333, %https://baijiahao.baidu.com/s?id=1588454914189589504&wfr=spider&for=pc
  \begin{abstract}
    The new HEVC standard enables a major advance in compression relative to its predecessors,  and its development was a large collaborative effort that distilled the collective knowledge of the whole industry and academic community into a single coherent and extensible design. This book collects the knowledge of some of the key people who have been directly involved in developing or deploying the standard to help the community understand the standard itself and its implications. A detailed presentation is provided for each of the standard’s fundamental building blocks and how they fit together to make HEVC the powerful package that it is. The compression performance of the standard is analyzed,  and architectures for its implementation are described.
  \end{abstract}

  \section{绪论}
    视觉是人类感知世界最重要的方式。中文说``眼见为实'', 英文强调``Seeing is believing'', 人们总是乐于接受所能看到的信息, 因此人类的科技一直不懈地致力于为人们提供更多、更好的用于``看''的信息。黑白电视的发明曾经给人们巨大的震撼, 很快, 人们对于色彩的天然渴望, 又促使彩色电视出现。进入数字时代之后, 数字视频更是紧随着IT技术的浪潮, 获得了非常迅速的发展。追求更高的清晰度, 是数字视频技术领域从未停止的步伐。如今, 各式各样的视频应用已经渗透到人类社会的各个领域, 可以说, 视频应用是现代人类社会运转的重要组成部分。
    \subsection{视频编码概述}
      视频包含的数据量十分庞大。以常见的标清电影视频格式(720p)来说, 未经压缩每秒需要处理的数据量为$1280\times 720(pixels/frame)\times 3(Byte/pixel)\times 30(frames/second)=79.1MB/s$! 远远超过了一般用户数据处理能力的极限, 更不用说4K视频以及流媒体应用了。
      因此, 视频编码技术是视频应用的一个十分重要的部分, 也称视频压缩, 目的是尽可能去除视频数据中的冗余, 减少压缩或编码后的数据量。所幸的是视频中包含大量的冗余信息, 通过去除掉这些冗余视频可以获得极高的压缩比。
    \subsection{数据压缩原理}
      由香农的信息论, 一个离散信源X可以最小平均码长$l=H(X)$信源熵来表示而不损失信息, 这种不失真的压缩方法称为无损压缩, 又称熵编码, 常见原理有Lempel-Ziv算法, Huffman编码, 算术编码, Golomb码, RLE(游程编码)等。但是熵编码一般难以达到5:1以上的压缩率, 故视频压缩都采用有损压缩方式。

      有损压缩的原理可以用率失真原理来解释。对一离散信源, 给定一定失真度D, 则可以以最小平均码长R(D)其率失真函数来表示。具体实现上就是对原信源引入难以察觉的失真来换取更小的信源熵, 从而得到更高的压缩率。例如人眼对图像的细节, 即图像中的高频成分不十分敏感, 则在压缩时丢掉部分高频信息可能并不会被人眼察觉。\textbf{如图。}此外, 即使压缩产生的失真能被人感知到, 但是如果这不会影响对视频内容的理解的话人们也通常愿意接受质量稍差的音、视频或者图像。在音视频和图像压缩算法中, 大量利用了人类的感知特性, 尽可能使压缩产生的失真发生在人不容易察觉到的地方。
    \subsection{视频编码标准}
      各种各样的视频应用从一开始就催生了多种视频编码方法。为了使编码后的码流能够在大范围内互通和规范解码, 从20世纪80年代起, 国际组织开始对视频编码建立国际标准。视频编码的国际标准通常代表这同时代最先进的视频编码技术。本文介绍的就是目前国际上最先进的视频编码标准H.265/HEVC。%值得注意的是, 视频编码标准仅规定了编码码流的语法语义和解码器,   因此符合某个视频编码标准的编码器是有很大自由度的

      \subsubsection{H.26x系列}
      H.261 混合编码(Hybrid Coding)始祖
      \subsubsection{MPEG系列}

    \subsection{H.265/HEVC简介}
      \subsubsection{标准化历程}
        近年来, 随着高清、超高请视频(分辨率达$4K\times 8K、8K\times 4K$)应用逐步走进人们的视野, 视频压缩技术收到了巨大的挑战。此外, 各式各样的视频应用也随着网络和存储技术的发展不断涌现。如今, 数字视频广播、移动无线视频、远程监测、医学成像和便携摄影等, 都已走进人们的生活。视频应用的多样化和高清化趋势对视频压缩性能提出了更高的要求。为此, 2010年4月VCEG和MPEG再次组建视频编码联合组(Joint Collaborative Team on Video Coding,  JCT-VC), 联手制定新一代视频编码标准-----H.265/HEVC。

        2013年6月7日, ITU-T网站上正式发布了H.265/HEVC标准, 该标准可以免费下载。2013年11月25日, ISO/IEC正式发布了H.265/HEVC标准。

        标准发布之后, 相关标准的进一步工作仍然在继续。JCT-VC现有的工作主要集中在就H.265/HEVC的扩展内容进行完善, 如更高的比特深度、4:2:2、4:4:4色度采样视频、可伸缩HEVC编码(Scalable HEVC, SHVC)和多角立体编码等。
      \subsubsection{编码框架}
        从根本上来说, H.265/HEVC视频编码标准的编码狂及并没有革命性的改变。类似与以往的国际标准, H.265/HEVC仍旧采用混合编码框架, 包括变换、量化、熵编码、帧内预测、帧间预测以及环路滤波等模块。但是, H.265/HEVC几乎在每个模块都引入了新的编码技术。
        \begin{enumerate}
          \item 帧内预测
          
            图像具有很强的空间相关性。通过已编码像素块来预测当前像素块有很好的去相关作用, 从而达到去除冗余的效果。
          \item 帧间预测
          
            视频具有很强的时间相关性。通过已编码帧来预测当前帧来获取块的运动信息可以达到去相干的作用, 从而去除冗余。
          \item 变换编码与量化
          
            变换编码通过对块进行频域变换把能量集中在低频区域, 量化把无限精度的数值以有限精度表达, 从而达到压缩效果。在HEVC中为了减少计算复杂度将变换编码与量化结合起来。该模块是编码中唯一有损的部分, 同时也是速率控制的关键。
          \item 环路滤波
          
            基于块编码后的图像会出现方块效应, 采用去方块滤波后可削弱其影响。对一信号采取有限频率逼近会出现振铃现象, 可通过SAO(样点自适应补偿)滤波来削弱。环路滤波目标即为削弱编码过程中预测、变换和量化等环节中引入的失真。
          \item 熵编码

            经各路编码后的变换系数、运动矢量、参数集等仍有压缩空间。实际存储或传输的数据为经过熵编码后的二进制流。
        \end{enumerate}

        除此之外, 为了适应不同的网络环境与视频应用, H.265/HEVC也采用了视频编码层(Video Coding Layer,  VCL)和网络适配层(Network Abstract Layer,  NAL)的双层架构;随着处理器多核架构的发展, 多核并行处理成为提高编解码能力的有效手段, H.265/HEVC在设计时充分考虑到了并行处理的必要性, 在数据单元划分与语法结构上均为其做了准备;为了在满足信道带宽和传输时延限制的情况下有效传输视频数据, 速率控制是实际编码器的很重要的组成成分。H.265/HEVC的官方编码器HM也提出了一些速率控制算法等, 本文中并不介绍, 具体可以参考\cite{hevc}。

      \section{编码结构}
        在基于块的视频混合编码架构中,每一张图像都被分解为不同的像素块,不同的像素块组合构成了不同的片(Slice)作为独立的解码单元。HEVC标准也继承了这一点。但在图像与块的划分上HEVC创新性地提出了基于四叉树的递归划分方式,能更灵活、高效地表示视频场景中的不同纹理细节、运动变化的视频内容或者视频对象。实验证实相较于上一代编码标准H.264 | MPEG-4 AVC,HEVC一半以上的比特率降低都归功于这种灵活的划分方式。

\subsection{HEVC编码结构简介}
HEVC标准以在以往获得很大成功的混合编码架构为基础。在该架构中,首先一帧的图像被分解成不同的块,而后每一块采用预测编码来消除其空间或者时间上的冗余。其中帧内预测以同一帧中已解码的像素块作为参考、帧间预测以已解码的其他帧作为参考。帧内预测利用了同一图像中临近块间的空间冗余进行压缩,帧间预测利用了大量的块运动信息来消除视频的时间冗余。无论是哪种方式,最终预测的误差,即预测像素与原始像素的残差经变换编码与标量量化被去相关,最终经过熵编码传输。图\ref{FW:pic1}以框图形式表示了整个编码流程。
\begin{figure}[H]
  \centering
  \includegraphics{pict/Block_diagram.png}
  \caption{Block diagram of an HEVC encoder with built-in decoder (gray shaded)}
  \label{FW:pic1}
\end{figure}

%宏块对比?
该图同时展示了HEVC特有的编码结构,这些将在后面的部分分别叙述。
首先我们看一下这个分块结构:HEVC中每一帧都被分解为同样大小不相交的方块,又称树形编码块(Coding Tree Block,CTB),每一个CTB作为四叉树编码结构的根节点,其大小由编码器指定。
CTB还可被进一步细分为编码块(Coding Block,CB),CB是编码器决定帧内预测与帧间预测进行编码的基本单元。每一帧划分为CTB与CTB划分为CB的方式将在章节\ref{FW:partSec1}里描述。
CB的划分方式在很大程度上与与CTB无关,具体方式将在章节\ref{FW:partSec2}里介绍。
变换编码依赖于CB的预测残差,但可在CB的基础上被进一步细分为变换编码块(Transform Block,TB),具体方式将在章节\ref{FW:partSec3}里介绍。
最后,章节\ref{FW:partEnd}提供了一些HEVC的实验数据,并将其与以前的编码标准做对比。

Slice层面的分割为并行处理提供了方便,将在并行处理部分讲解。
%分片<-并行处理

\subsection{块分割}
自H.261以来,所有ITU-T与 ISO/IEC视频编码标准都使用混合编码架构,而不同标准间的区别则主要体现在对同一块样本块不同标准可供选择的编码模式集的不同。一方面,所选编码模式决定了该样本块是使用帧内预测还是帧间预测;另一方面,解码器需要知道一样本块是怎么被分解为帧内/帧间预测块的。通常来说被预测的块还需要传递其参数,对于帧内预测需要传递预测模式,帧间预测则需要传递运动矢量信息。

为了给编码器与解码器的开发者足够的自由,同时保证不同厂家生产的设备的互操作性,视频编码标准仅规定了编码码流的语法语义与解码过程,具体编码过程则未作要求。因此,编码效率在很大程度上依赖于决定编码语义元素的算法,其中包括编码模式的选择、预测参数、量化参数与已量化变换矩阵索引等。一个简单且行之有效的方法是拉格朗日率失真优化(Rate–distortion optimization,RDO)。在这种方式中,参数$p^*$的选择由在可选参数集$\mathscr{A}$中最小化损失函数$D$与编码比特数$R$的加权和来决定,
\begin{equation}
  p^*=\arg\min_{\forall p \in \mathscr{A}}D(p)+\lambda \cdot R(p)
\end{equation}
其中拉格朗日参数$\lambda$为一常数,用来权衡失真$D$与比特数$R$,因此视频质量与码率被兼顾。

一个混合编码可达到的编码效率取决于很多设计准则,像是内插滤波器的设计,熵编码的效率,以及环路滤波方法,然而两代编码标准的性能提升最关键在于可供选择的块编码方式的增加,这体现在帧间预测时更高精度的运动矢量、更灵活的帧编码顺序、更多的参考帧的获取,帧内预测时更多的预测模式,以及更多的运动矢量预测器(?),更多的变换块大小以及运动补偿块大小等。

然而也不是分块与预测方式越多越好。考虑一给定的样本块,我们可以把它分为更小的子块选取更佳的预测参数来获得更小的误差,代价则是更高的的码率;同样若不再细分,则随码率的下降误差也会增加,具体哪种更好取决于待编码块。当可供选择的分块方式增加时,大体上我们需要更多的比特数来表示所选的模式,同时编解码复杂对也会相应增加。因此,标准的设计需要综合考虑这些方面。

由于个人电脑计算能力的提升,新的视频编码标准大体上会支持更多的编码选项。在HEVC标准设计时,考虑到高清(High Definition,HD)与超高请(Ultra High Definition,UHD)视频的需求,HEVC支持了更大的编码块用以进行预测与变换;同时为了不遗漏细节,小的编码块也是十分重要的。这两种截然相反的目标被HEVC以一种开创性的,同时简单而有效的递归四叉树划分法解决了。除此之外,这种四叉树划分方式也支持一种快速最优化算法\cite{Chou1989OptimalPW}来计算拉格朗日率失真代价。

\subsubsection{CTU与CTB}
  以往ITU-T与 ISO/IEC提出的编码标准中,每一帧都被分解为宏块,每个宏块包含大小为$16\times 16$的亮度采样块。色度采样块由视频采样率决定,对于4:2:0采样率的视频其包含两块$8\times 8$的色度采样块。宏块为编解码处理的基本单元,对每一个宏块均需要确定其预测方式。

  尽管目前为止H.262 | MPEG-2与H.264 | MPEG-4 AVC标准也被用于保存与传输高清(HD)视频内容,其最初设计目标是分辨率在QCIF($176\times 144$)到标清($720\times 480,720\times 576$)的视频。由于分辨率可高达$3840\times 2016$与$7680\times 4320$的HD与UHD视频的兴起,HEVC最初设计时就考虑到了高分辨率视频。在如此高的分辨率下视频中会有大块平坦区域与大块运动矢量一致的区块,因此采用更大的分块大小能明显减少视频容量。同时HEVC标准也被设计用于为所有现有视频内容提供优于上一代视频编码标准H.264 | MPEG-4 AVC的压缩效率,因此也必须设计出小尺寸的编码块。基于此,HEVC提供了灵活的基于四叉树的分块方式。

  在HEVC中,为使图像总体整体CTB数目一致,每一分块均包含一个亮度CTB与两个色度CTB。每一亮度采样块与其对应的色度采样块,加上其语法元素构成了一个树形编码单元(Coding Tree Unit,CTU),作为HEVC的基本处理单元,概念上与宏块相似。亮度CTB大小为$2^N\times 2^N$,对于4:2:0采样率视频来说每一色度CTB大小为$2^{N-1}\times 2^{N-1}$。N可取值4,5或6,分别代表的CTU大小为$16\times 16,32\times 32\textrm{与}64\times 64$,并于一开始在序列参数集(Sequence Parameter Set,SPS)中传输。图\ref{FW:pic2}表示将一分辨率为$1280\times 720$的视频分解为$16\times 16$宏块与$64\times 64$的CTU的结果。
  \begin{figure}[H]
    \centering
    \includegraphics{pict/Partition.png}
    \caption{图像划分方式:(a)$16\times 16$宏块;(b)$64\times 64$ CTU。可见对于该图像来说$16\times 16$的分块方式明显不如$64\times 64$的方式编码来得有效率}
    \label{FW:pic2}
  \end{figure}
  
  编码器可自主选择大的CTU用以适应更高分辨率视频或者更高的编码效率,或者选择更小的CTU用于适应低分辨率视频或者更好的保真度。

\subsubsection{CTU与CU}
  在以往的ITU-T与 ISO/IEC视频编码标准中,宏块为预测的基本单元,每一宏块均需确定其预测方式为帧内预测还是帧间预测,在不同的预测方式中,宏块需再分解为更小的子块来分别进行预测。

  在上一代视频编码标准H.264 | MPEG-4 AVC中,帧内预测有三种分块方式,分别为$Intra-4\times 4$,$Intra-8\times 8$与$Intra-16\times 16$;帧间预测有四种分块方式,分别为$Inter-16\times 16$,$Inter-16\times 8$,$Inter-8\times 16$与$Inter-8\times 8$,图\ref{FW:pic3}表示了这些方式。
  \begin{figure}
    \centering
    \includegraphics{pict/Macro.png}
    \caption{Macroblock partitioning modes supported in the High profile of H.264 | MPEG-4 AVC for inter-picture coding (top line) and intra-picture coding (bottom line). If the $Inter-8\times 8$ is chosen, the $8 \times 8$ sub-macroblocks can be further partitioned into $8\times 4$, $4\times 8$, or $4\times 4$ blocks}
    \label{FW:pic3}
  \end{figure}
  每种预测方式中一块的预测参数由其相邻块的已解码信息判断。

  在HEVC中,CTU的大小可以达到$64\times 64$,因此宏块的方式明显不可取。一方面基于CTU的预测方式选择过于粗糙,难以很好地重构原图像;另一方面如果要支持如H.264 | MPEG-4 AVC宏块形式的块分割方法的话语法会变的很复杂,无法改变的块大小也很不适合于视频压缩中。

  为了解决这些问题,HEVC将每一个CTU以四叉树的方式进一步分解为编码单元(Coding Unit,CU),如图\ref{FW:pic4}所示。
  \begin{figure}
    \centering
    \includegraphics{pict/CTU.png}
    \caption{Example for the partitioning of a $64\times 64$ coding tree unit (CTU) into coding units (CUs) of $8 \times 8$ to $32 \times 32$ luma samples. The partitioning can be described by a quadtree, also referred to as coding tree, which is shown on the right. The numbers indicate the coding order of the CUs}
    \label{FW:pic4}
  \end{figure}
  与CTU类似,每个CU包含一个亮度采样块与两个色度采样块与其语法元素,作为决定预测方式的基本单元,而在预测编码与变换编码中一个CU还可进一步被分解为更小的预测单元(Prediction Unit,PU)与变换单元(Transform Unit)。

  在CTU层面上,一个CTU是否被分解由标志位\texttt{split\_cu\_flag}决定,而其分解后的每一块是否被分解由另一标志位\texttt{split\_cu\_flag}决定。当无待分解块时,分解结束。CU的最小大小在序列参数集(Sequence Parameter Set,SPS)中指定,其大小在$8 \times 8$与CTU大小之间。一般编码配置充分利用了CU大小的可变性,其尺寸在$8 \times 8$与$64 \times 64$之间。

  CU编码采用深度优先顺序,或者称为Z型扫描顺序。使用这种编码方法可以保证除了在左上边缘处的CU,其余CU编码时其左侧或上侧的CU已经编码,则其样值与预测参数可以拿来预测当前CU的编码参数。

  整体上来说,CU与以前编码标准中的宏块很相似,但是CU大小可变,这就给予了HEVC更高的灵活性。

\subsubsection{PB与PU}
  对每一CU,需要决定其预测方式;而一旦其预测方式确定了,其编码参数也要相应地确定下来。
  
  对于帧内预测来说,共有35种空域预测模式。如果CU尺寸与SPS中规定的最小CU尺寸一致,亮度CB可进一步被分解为四个同样大小的子块,需要分别传输其预测模式。而色度CB与其对应CU尺寸无关,两色度CB采用同种预测方式。色度CB可选五种预测方式,其中一种与亮度CB一致或在亮度CB传输四个预测模式时为其第一个。
  %最小CU再分原因
  实际的帧内预测并不一定以已确定预测模式的编码块为单位进行。实际上每一编码块有可能被分解为更多的变换块(Transform Block,TB),而残差的计算基于已重建的像素块进行,如图\ref{FW:pic5}所示,
  \begin{figure}
    \centering
    \includegraphics{pict/Prediction1.png}
    \caption{Illustration of the horizontal intra prediction of a selected sample inside an $8 \times 8$ coding block with $4 \times 4$ transform blocks, if the intra prediction is applied on the basis of coding blocks (a) or transform blocks (b)}
    \label{FW:pic5}
  \end{figure}
  当变换块增大时,由于预测点距离增大误差一般会更大,但对应的比特率也会减少,因此这种可选特性提供了一种权衡质量与比特率的手段。
  
  如果一CU采用帧间预测方式,其亮度与色度CB可被进一步分解为预测块(Prediction Block,PB),每一预测块共享同一运动参数。运动参数包括运动候选(1或2)、参考图像索引以及每种运动候选的运动矢量。某一CU的亮度CB与其两色度CB采用同种划分方式。亮度PB,其对应的两色度PB,与其语法元素构成了预测单元(Prediction Unit,PU),实际传输PU的预测参数。

  HEVC支持8种PU划分方法,如图\ref{FW:pic6}所示,
  \begin{figure}
    \centering
    \includegraphics{pict/Prediction2.png}
    \caption{Supported partitioning modes for splitting a coding unit (CU) into one, two, or four prediction units (PU). The $(M/2)\times(M/2)$ mode and the modes shown in the bottom row are not supported for all CU sizes}
  \end{figure}
  一个CU可以整体作为一个PU来进行预测,也可划分为子块来进行,其中$(M/2)\times(M/2)$方式仅当最小CU尺寸大于$8\times 8$时可用,非对称形式仅对$8\times 8$尺寸以上的CU可用,因此帧间预测的最小尺寸为$8\times 4$与$4\times 8$。

  如此多的区块划分方式为提高编码效率提供了可能,但是编码器需要遍历所有方式,运算量明显很大。同时为表示这么多种预测方式所对应的语法元素最终甚至可能降低编码效率。为此,可在SPS中指定可选的分块模式集来权衡质量与码率。

\subsubsection{残差四叉树(RQT),TB与TU}
  如上一章节所述,在预测残差的变换编码中,一个CB可被分解为多个变换块(Transform block,TB),这种分解基于一种叫残差四叉树(Residual Quadtree,RQT)的结构递归进行。在每个RQT中,CB为该树的跟节点,每个TB为位于该树的叶子节点上,图\ref{FW:pic6}为将一$64\times 64$亮度CTB分解为亮度CB与亮度TB的过程。
  \begin{figure}
    \centering
    \includegraphics{pict/TB.png}
    \caption{Example for the partitioning of a $64 \times 64$ luma coding tree block (black) into coding blocks (blue) and transform blocks (red). In the illustration on the right, the blue lines show the corresponding coding tree with the coding tree block (black square) at its root and the coding blocks (blue circles) at its leaf nodes; the red lines show the non-degenerated residual quadtrees with the transform blocks (red circles) as leaf nodes. Note that the transform blocks chosen identical to the corresponding coding blocks are not explicitly marked in this figure. The numbers indicate the coding order of the transform blocks}
    \label{FW:pic6}
  \end{figure}
  色度CB基于同样的结构被分解为色度TB,但有一个例外,我们将在后面提到。
  
  允许不同大小的变换块给我们以不同尺度分析空频特性的可能:更大的变换块有更高的频率分辨率,然而更小的变换块有更高的空间分辨率,这两者的权衡可在编码器层面控制。

  每个RQT有三个参数:最大深度$d_{max}$、最小变换块尺寸$n_{min}$与最大变换块尺寸$n_{max}$,并与SPS中传输。后两者可取值2到5,代表的变换块尺寸为$4\times 4$到$32\times 32$。最大深度$d_{max}$限制了该RQT的深度,如$d_{max}=1$时,一个亮度CB可作为一个亮度TB或被分解为4个亮度TB,但不可再分。需要注意的是有时变换块尺寸被隐含在参数中,如即使$d_{max}=0,n_{max}=5$,对于一$64\times 64$的亮度CB,仍必须将其分为四个$32\times 32$的亮度TB。

  若去相干变换作用于多个预测块,变换残差经常包含预测边界,这会导致高频分量能量增加从而降低编码效率。由于这个原因,在$d_{max}=0$时HEVC也包含了一个隐式分割的条件。若$d_{max}=0$,且一个CU采用帧间预测方式,同时一个CU被分解为多个PU,此时亮度CB总是被分解为四个亮度TB。当$d_{max}>0$且CU采用帧间预测方式时,RQT分割与PU分割无关,因此一个TB可能包含多个PB。虽然这样可能会降低该CB的编码效率,但是同时别的CB的编码效率会增加。研究表明\cite{23}当采用这种方式时其Bjøntegaard Delta bit rate (BD rate)会增加$0.4-0.7\%$。帧内预测则不同,一个TB不可跨越多个预测区块,则当一个亮度CB尺寸等于最小CB尺寸,且其被分解为四个预测块传输不同的预测参数时,该CB必被分解为四个TB。同时分解得到的四个TB可能再次被分解。

  对每个CU,至多一个RQT语法元素被同时传输,该RQT同时决定了所有颜色分量的划分方式。但有一个例外,对于4:2:0采样率的视频来说,若亮度TB尺寸为$4\times 4$时,色度TB不被再分。一个尺寸大于$4\times 4$的亮度TB,或者四个尺寸为$4\times 4$的亮度TB,与其伴随的两色度TB加上其他语法元素组成了一个变换单元(Transform Unit,TU)。

  由于RQT位于CU中,必须对每个CU传输其语法元素。例如当CU传递完其预测模式,PU分块以及PU相关语法元素后,若其变换系数等级不为0,语法元素\texttt{split\_transform\_flag}会被传输来表示其是否为叶子节点。当隐式推测发生时,则由译码器推测其具体数值。

  除了RQT的具体结构外,也需要知道对于一TB或者整个CU来说变换系数等级是否为0。对采用帧间预测的CU来说,标志位\texttt{rqt\_root\_cbf}用来表示是否至少有一个TB的等级不为0。当其为1时如上所述传输,否则不再传输残差矩阵,其数值被视作0。该语法元素对于低码率编码以及可精确预测的区域有十分重要的意义。对skip模式的CU来说,标志位\texttt{cu\_skip\_flag}被置为1,此时没有残差需要传输,对应的也没有RQT语法元素需要传输。然而对于帧内预测的CU来说,\texttt{rqt\_root\_cbf}总被视为1,因此总可以认为至少一个TB的等级不为0。

  更进一步说,当\texttt{rqt\_root\_cbf=1}时,对每个亮度TB与其对应的两个色度TB也需要传输另一个cbf为。亮度由\texttt{cbf\_luma}表示,色度标志位\texttt{cbf\_cb}与\texttt{cbf\_cr}与\texttt{split\_transform\_flag}被交错编码。这种编码方式在有一个或所有色度TB残差均为0,而亮度TB残差不为0时编码效率进一步增加。Details\cite{24,30}。

  随RQT与其他编码树分解深度的增加,可供选择的分解方式也呈双指数速率$2^{4^{d-1}}$增加。然而,文章\cite{25}指出,通过采用通用BFOS算法\cite{5},率失真下的最优分块方法并不一定需要穷举才能得到。实际上不采用early termination strategy的话,运算复杂度与$(4^d-1)/3$成正比,为了进一步降低运算复杂度,可使用heuristic early-pruning techniques\cite{25,38}。
  
  文章\cite{38}针对RQT分块结构提出了一种算法,思想是当所有未量化变换系数低于合理选取的量化器步长阈值时,进一步细分可被终止。采用这种策略,编码器运行时间可减少$5-15\%$,而编码器效率仅有很少损失。当RQT更深时,耗时会减少更多。Details\cite{38}。
      \section{预测编码}
        预测编码是视频编码的核心技术之一。对于视频信号来说,   一幅图像邻近像素间有较强空间相关性,   相邻图像间有很强时间相关性。因此,   先进的视频编码常采用帧内预测和帧间预测的方式,   去除空域和时域的相关性,   之后编码器对预测后的残差进行变换,   量化,   熵编码,   可以大幅提高编码效率。

原理用信息论可以解释如下:对于独立信源X与Y, 可以证明$H(X+Y)\geq max(H(X),H(Y))$。对于帧内预测假设图像由某一纹理X与随机残差Y构成, 而针对于帧间预测假设图像由其参考图像X与随机残差Y构成。对整体信源进行压缩信息量为$H(X+Y)$, 若选择一合适纹理进行编码则原信源信息为$\frac{1}{n}H(X)+H(Y)$, 由上述不等式当$n\to\infty$时必有去相关的编码模式压缩效率更高。

预测流程如下:
\begin{enumerate}
    \item 以同一帧图像内的临近像素作为参考,   计算预测值Xp
    \item 原始值X和预测值Xp的差值d,   被传递到解码端
    \item 解码端接收到差值d,   将其与预测值Xp相加,   就得到了``原始值''X',   X'=Xp+d
\end{enumerate}

整个过程可以可以简化为下图\ref{PC:1}:
\begin{figure}[H]
    \centering
    \includegraphics[width=.6\textwidth]{pict/PC/Intra/1.png}
    \caption{流程图}
    \label{PC:1}
\end{figure}
该方法同时适用于帧内预测和帧间预测。
\subsection{帧内预测}
\subsubsection{简述}
视频序列中的每帧图像,   局部会有描述同一物体的情况,   而描述同一物体的相邻像素间就会有相关性,   并不是绝对独立的。
    
这时,   用周围已编码部分表示(预测)当前所需编码的部分,   这种在空间域上进行的预测编码算法,   可以除去相邻块之间的空间冗余度,   取得更为有效的压缩。

对于存在前后相关性的信息,   预测编码是一种非常简便且有效的方法。此时预测编码输出的不再是原始的信号值,   而是信号的预测值与实际值的差。预测编码如此设计的出发点在于,   由于前后存在相关性,   相邻信号存在大量相同或相近的现象,   通过计算其差值,   可以减少大量保存与传输原始信息的数据体积。

如图\ref{PC:2}所示
\begin{figure}[H]
    \centering
    \includegraphics[width=.6\textwidth]{pict/PC/Intra/2.png}
    \caption{示例图片\cite{万帅2014新一代高效视频编码}}
    \label{PC:2}
\end{figure}
左图待预测区域内容较为平坦,   可用参考像素取平均的DC模式,   而右图纹理呈水平状排列,   可以采用水平的预测模式。
\subsubsection{帧内预测技术(基于H.264标准的亮度信号预测)}
%\subsubsubsection{基于H.264标准的亮度信号预测}

\paragraph{4x4亮度块,   九种预测模式}

在H.264中用九种预测模式,   分别是:垂直预测,   水平预测,   DC预测,   以及五种不同倾斜方向的预测模式。图\ref{PC:3}中4×4亮度块的上方和左方像素A--M为已编码和重构像素,   用作编解码器中的预测参考像素。a--p为待预测像素,   利用A--M值和9种模式实现。
\begin{figure}[H]
    \centering
    \includegraphics{pict/PC/Intra/33.png}
     \caption{$4\times4$亮度块,   九种预测模式\cite{万帅2003新一代视频压缩标准}}
      \label{PC:3}
\end{figure}
\begin{figure}[H]
   \centering
   \includegraphics[width=.5\textwidth]{pict/PC/Intra/4.png}
   \caption{samples}
   \label{PC:4}
\end{figure}
\paragraph{16x16帧内预测模式}
如图\ref{PC:5},   帧内预测有垂直,   水平,   直流DC和planar四种模式。
\begin{figure}[H]
   \centering
   \includegraphics{pict/PC/Intra/5.png}
   \caption{16x16帧内预测模式}
    \label{PC:5}
\end{figure}

\subsubsection{选择预测模式的标准:XP和X’差值越小越好}
为了选择出合适的帧内预测模式,   H.264/AVC采用了拉格朗日率失真优化(RDO)进行模式选择。它为每一种 模式计算拉格朗日代价:
\begin{equation}
J=D+\lambda\times R
\end{equation}
其中,   D表示当前预测模式下的失真,   R表示编码当前预测模式下所有信息(如变换系数,   模式信息,   宏块划分方式等)所需比特数,   为拉格朗日因子。

需要说明的是,   最优的预测模式不一定残差最小,   而是残差信号经其他编码模块(变换,   量化,   熵编码等)后最终编码性能最优。
\subsubsection{H.265/HEVC的精进之处}
H.265是新的编码协议,   也即是H.264的升级版。H.265标准保留H.264原来的某些技术,   同时对一些相关的技术加以改进。新技术使用先进的技术用以改善码流、编码质量、延时和算法复杂度之间的关系,   达到最优化设置。

\paragraph{H.265帧内预测概述}

在H.265中,   $4\times 4$块预测模式从9种增加至35种,   35种预测模式是在PU的基础上定义的,   而具体帧内预测过程的实现则是以TU为单位的。

H.265/HEVC帧内预测可分为以下3个步骤:
\begin{itemize}
\item 判断当前TU相邻参考像素是否可用并做相应的处理
\item 对参考像素进行滤波
\item 根据滤波后的参考像素计算当前TU的预测像素值
\end{itemize}

\paragraph{相邻参考像素的选取}

如图\ref{PC:6},   当前的TU大小为NxN,   其参考像素按区域可分为5部分:左下(A)、左侧(B)、左上(C)、上方(D)和右上(E), 一共4N+1个点。若当前TU位于图像边界,   则相邻参考像素可能会不存在或不可用。另外,   在某些情形下A或E所在的块可能尚未进行编码,   此时这些参考像素也是不可用的。
\begin{figure}[H]
    \centering
    \includegraphics[width=.6\textwidth]{pict/PC/Intra/6.png}
    \caption{参考像素的选取}
    \label{PC:6}
\end{figure}
当参考像素不存在或不可用时,   H.265/HEVC标准会使用最邻近的像素进行填充。例如,   若区域A的参考像素不存在,   则区域A所有参考像素都用区域B最下方的像素进行填充;若区域E的参考像素不存在,   则区域E所有参考像素都用D最右侧的像素进行填充。需要说明的是,   若所有参考像素都不可用,   则参考像素都用固定值填充,   该固定值大小为
\begin{equation}
R=1\leq (bitdepth-1)
\end{equation}
\paragraph{参考像素的滤波}

由于帧内预测过程中,   边缘会发生突变,   这时就需要滤波。滤波的目的在于提升帧内预测的像素块的视觉效果,   减小边缘可能产生的突变感。是否对参考像素进行滤波取决于帧内预测模式和预测像素块的大小。 
滤波一般是通过一个三抽头滤波器实现,   三抽头滤波器系数为[0.25, 0.5, 0.25]。

对于DC模式以及4*4大小的TU都不需要对参考像素滤波处理,   但是其他部分模式由于特殊需要,   要进行滤波处理,   这里不赘述。
\paragraph{预测像素的计算}

与h.264/avc相比,   h.265/hevc增加使用了左下方块的边界像素作为当前块的参考。这是由于h.264/avc以固定大小的宏块为单元进行编码,   在对当前块进行帧内预测时,   其左下方块很有可能尚未进行编码,   无法用于参考;而h.265/hevc四叉树形的编码结构使得这一区域成为可用像素。此外,   这一区域像素的使用也提供了更多可能的预测方向,   在某些情形下(如倾斜向上方向的纹理等)能够大幅度提高预测精度。

\subsubsection{HEVC帧内预测模式}
H.265/HEVC亮度分量帧内预测支持5种大小的PU: 4x4,  8x8,  16x16,  32x32,  64x64
每一种大小的PU都有35种预测模式:

\paragraph{Planar模式}

Planar模式是由H.264/AVC中的Plane模式发展而来的,   它适用于图像值缓慢变化的区域。Planar模式使用水平和垂直方向的两个线性滤波器,   并将二者的平均值作为当前块像素的预测值。
\begin{figure}[H]
    \centering
    \includegraphics{pict/PC/Intra/7.png}
     \caption{Planar模式}
      \label{PC:7}
\end{figure}
\begin{figure}[H]
    \centering
    \includegraphics{pict/PC/Intra/8.png}
     \caption{Planar模式}
      \label{PC:8}
\end{figure}
算法如下:
\begin{align}
P_{x, y}^H=(N-x)R_{0, y}+x\times R_{n+1, 0}\\
P_{x, y}^V=(N-y)R_{0, x}+y\times R_{0, N+1}\\
P_{x, y}=(P_{x, y}^H+P_{x, y}^V+N)\div 2N    
\end{align}
\paragraph{DC模式}

DC模式适用于大面积平坦区域,   其做法与H.264/AVC基本相同。当前块预测值可由其左侧和上方(注意不包含左上角、左上方和右上方)参考像素的平均值得到。
\begin{figure}[H]
    \centering
    \includegraphics{pict/PC/Intra/9.png}
    \caption{DC模式}
    \label{PC:9}
\end{figure}
算法如下:
\begin{equation}
DCvalue=\sum_{x=1}^NR_{x, 0}+\sum_{y=1}^NR{0, y}\div2N
\end{equation}
左上角像素:
\begin{equation}
P_{1, 1}=(R_{1, 0}+R_{0, 1})\div 2
\end{equation}
第一行像素:
\begin{equation}
{P_{x, 1}=(R_{x, 0}+3\times DcValue)\div 4}
\end{equation}
第一列像素:
\begin{equation}
P_{1, y}=(R_{0, y}+3\times DcValue)\div4
\end{equation}
其他像素:
\begin{equation}
P_{x, y}= DcValue
\end{equation}
\paragraph{三十三种角度模式}

H.264/AVC使用了8中不同的预测方向(4x4大小),   H.265/HEVC则进一步细化了这些预测方向,   规定了33种角度预测模式,   以更好地适应视频内容种不同方向的纹理。
下图\ref{PC:10}给出了33种角度模式的具体方向,   其中V0(模式26)和H0(模式10)分别表示为垂直和水平方向,   其余模式的预测方向都可以看成再垂直或水平方向上做了一个偏移,   该偏移角的大小可由模式下方的数字计算得出。
\begin{figure}[H]
    \centering
    \includegraphics{pict/PC/Intra/10.png}
     \caption{$\theta=arctan(\frac{x}{32})$}
      \label{PC:10}
     
\end{figure}

$\theta$ 为正表示预测方向向左偏移,   $\theta$ 为负表示预测方向向右偏移;对于水平类模式,   $\theta$ 为正表示预测方向向上偏移,   $\theta$ 为负表示预测方向向下偏移
算法如下:

33种角度模式可分为水平类模式(2--17)和垂直类模式(18--34)

下面以垂直类模式为例给出预测像素计算值:
首先,   对于部分垂直类模式,   即offset(M)<0的模式,   既需要用到左侧像素又需要用到上层像素,   带来了编码的复杂性。有没有一种方法能简化这种复杂度呢,   答案是有的。
这里,   我们就需要将左侧像素映射到上层,   如图\ref{PC:11}所示:

\begin{figure}[H]
    \centering
    \includegraphics[width=.8\textwidth]{pict/PC/Intra/11.png}
    \caption{映射示意图}
    \label{PC:11}
\end{figure}
建立一维数组Ref,   

Offset(M)<0时,   

$$ Ref(x)=\left\{
\begin{aligned}
 R_{0, x},  x\geq0\\
 R_{y(x), 0}, x<0
\end{aligned}
\right.
$$


其中
\begin{equation}
y(x)=round(\frac{32\times x}{offset(M)})
\end{equation}
Offset(M)>0时,   
\begin{equation}
Ref(x)=R_{0, x}    , x=0, 1, 2, 3, ………2N
\end{equation}
接下来建立当前像素对应参考像素在ref里的位置:\\
\begin{equation}
Pos=floor((x\times offset(M))\div32)
\end{equation} \\
加权因子:$W=(x\times offset(M)\&31$,   其中\&表示按位与运算,   可以理解为提取后四位。
最后计算当前像素值:
\begin{equation}
P_{x, y}=((32-w)\times ref[y+pos]+w\times ref[y+pos+1])\div32
\end{equation}
需要注意的是,   对于模式26(垂直模式),   预测像素值为:
\begin{equation}
\left\{
\begin{aligned}
  P_{x, y}&=R_{0, y}\\
 P_{x, y}&=R_{x, 0}\div2+R_{0, 0}
\end{aligned}
\right.
\end{equation}
35种预测模式实现效果图如下\ref{PC:12}:
\begin{figure}[H]
    \centering
    \includegraphics[width=.8\textwidth]{pict/PC/Intra/texture.png}
    \caption{效果图}
    \label{PC:12}
\end{figure}
        \subsection{帧间预测}%2
\subsubsection{基本原理}%2.1
两帧之差的物体运动一般是刚体的平移运动,   位移量不大,   因此可将活动图像分成若干块或宏块,   并设法搜索出每个块或宏块在邻近帧图像中的位置,   并得出两者之间的空间位置的相对偏移量,   得到的相对偏移量就是通常所指的运动矢量,   得到运动矢量的过程被称为运动估计。视频压缩的时候,   只需保存运动矢量和残差数据就可以完全恢复出当前块。
\begin{figure}[H]
  \centering
  \includegraphics[width=.8\textwidth]{pict/PC/Inter/1.png} %1.png是图片文件的相对路径
  \caption{残差帧(没有进行运动补偿)} %caption是图片的标题
  \label{PC:img} %此处的label相PC:当于一个图片的专属标志,   目的是方便上下文的引用
\end{figure}
H.264 编码器为帧的每个部分选择了最佳分割尺寸,   使传输信息量最小,   并将选择的分割加到残差帧上。在帧变化小的区域(残差显示灰色),   选择$16\times 16$ 分割;多运动区域(残差显示黑色或白色),   选择更有效的小的尺寸。

\subsection{基本概念}%2.2
\subsubsection{帧的分类与分组}%2.2.1
当通过宏块扫描与宏块搜索发现相邻几帧的关联度非常高时,   这几帧就可以划分为一组。其算法是:在相邻几幅图像画面中,   一般有差别的像素只有10\%以内的点, 亮度差值变化不超过2\%,   而色度差值的变化只有1\%以内,   我们认为这样的图可以分到一组。在这样一组帧中,   经过编码后,   我们只保留第一帖的完整数据,   其它帧都通过参考上一帧计算出来。我们称第一帧为IDR/I帧,   其它帧我们称为P/B帧,   这样编码后的数据帧组我们称为GOP。
\begin{itemize}
  \item I帧(intra picture):关键帧,   采用帧内压缩技术。
  \item P帧:向前参考帧,   在压缩时,   只参考前面已经处理的帧。采用帧间压缩技术。
  \item B帧:双向参考帧,   在压缩时,   它即参考前而的帧,   又参考它后面的帧。采用帧间压缩技术。
  \item GOP:两个I帧之间是一个图像序列,   在一个图像序列中只有一个I帧。如\ref{PC:GOP示意图}所示。
\end{itemize}

\begin{figure}[H]
  \centering
  \includegraphics[width=.8\textwidth]{pict/PC/Inter/2.png} %1.png是图片文件的相对路径
  \caption{GOP示意图} %caption是图片的标题
  \label{PC:GOP示意图} %此处的label相PC:当于一个图片的专属标志,   目的是方便上下文的引用
\end{figure}
在通常的场景中,   编解码器编码一个I帧,   然后向前跳过几个帧,   用编码I帧作为基准帧对一个未来P帧进行编码,   然后跳回到I帧之后的下一个帧。编码的I帧和P帧之间的帧被编码为B帧。之后,   编码器会再次跳过几个帧,   使用第一个P帧作为基准帧编码另外一个P帧,   然后再次跳回,   用B帧填充显示序列中的空隙。这个过程不断继续,   每12到15个P帧和B帧内插入一个新的I帧。例如,   \ref{PC:zch3}给出了一个典型的视频帧序列:
\begin{figure}[H]
  \centering
  \includegraphics[width=.8\textwidth]{pict/PC/Inter/3.png} %1.png是图片文件的相对路径
  \caption{帧序列编码顺序} %caption是图片的标题
  \label{PC:zch3} %此处的label相PC:当于一个图片的专属标志,   目的是方便上下文的引用
\end{figure}

\subsubsection{宏块与宏块的划分}%2.2.2
一个编码图像通常划分成若干宏块组成,   一个宏块由一个$16\times 16$ 亮度像素和附加的一个$8\times 8 $Cb和一个$8\times 8 $Cr 彩色像素块组成。每个图象中,   若干宏块被排列成片的形式。

每个宏块($16\times 16$ 像素)可以4 种方式分割:一个$16\times 16$,   两个$16\times 8$,   两个$8\times 16$,   四个$8\times 8。$其运动补偿也相应有四种。而$8\times 8 $模式的每个子宏块还可以四种方式分割:一个$8\times 8$,   两个$4\times 8 $或两个$8\times 4 $及4个$4\times 4$。

这些分割和子宏块大大提高了各宏块之间的关联性。这种分割下的运动补偿则称为树状结构运动补偿。

每个分割或子宏块都有一个独立的运动补偿。每个MV 必须被编码、传输,   分割的选择也需编码到压缩比特流中。对大的分割尺寸而言,   MV 选择和分割类型只需少量的比特,   但运动补偿残差在多细节区域能量将非常高。小尺寸分割运动补偿残差能量低,   但需要较多的比特表征MV 和分割选择。分割尺寸的选择影响了压缩性能。整体而言,   大的分割尺寸适合平坦区域,   而小尺寸适合多细节区域。

H.264 编码器为帧的每个部分选择了最佳分割尺寸,   使传输信息量最小,   并将选择的分割加到残差帧上。在帧变化小的区域(残差显示灰色),   选择$16\times 16$ 分割;多运动区域(残差显示黑色或白色),   选择更有效的小的尺寸。

\begin{figure}[H]
  \centering
  \includegraphics[width=.8\textwidth]{pict/PC/Inter/4.png} %1.png是图片文件的相对路径
  \caption{H.264/AVC帧间预测块划分方式} %caption是图片的标题
  \label{PC:H.264/AVC帧间预测块划分方式} %此处的label相PC:当于一个图片的专属标志,   目的是方便上下文的引用
\end{figure}
相比于H.264而言,   灵活的块划分技术给H.265带来了很高的性能提升,   在参考软件中利用递归的方式实现了块的四叉树划分。H.265标准中对于编码单元有四个概念:编码树单元(CTU),  编码单元(CU),  预测单元(PU),   变换单元(TU)。

划分关系为:CTU可以四叉树划分为四个CU也可以不划分,   这是根据率失真代价决定的,    一个CU可以划分成多个PU,   H.265有8种划分模式,   具体选哪一种,   在划分中是根据率失真代价决定的。变换单元TU是在CU的基础上划分的,   跟PU没有任何关系,   采用四叉树划分方式,   具体划分有率失真代价决定。

\subsubsection{运动估计与运动估计的准则}%2.2.3
一般的运动估计方法如下:设t 时刻的帧图像为当前帧$f(x,   y)$,    $t’$时刻的帧图像为参考帧$f’(x,   y)$,   参考帧在时间上可以超前或者滞后于当前帧,   如图\ref{PC:前向和后向运动估计}所示,   当$t’<t$ 时,   称之为后向运动估计,   当$t’>t$ 时,   称之为前向运动估计。当在参考帧$t’$中搜索到当前帧$t$ 中的块的最佳匹配时,   可以得到相应的运动场$d(x;t, t+\Delta t)$,   即可得到当前帧的运动矢量。
\begin{figure}[H]
  \centering
  \includegraphics[width=.8\textwidth]{pict/PC/Inter/5.png} %1.png是图片文件的相对路径
  \caption{前向和后向运动估计} %caption是图片的标题
  \label{PC:前向和后向运动估计} %此处的label相PC:当于一个图片的专属标志,   目的是方便上下文的引用
\end{figure}
设前一帧搜索区为(M+2Wx,  N+2Wy),   当前帧块与前一帧块的位移为$d(i,  j)$,   在搜索区中,   如能找到与当前帧块匹配的前一帧块,   则该$d(i,  j)$即为所需的运动矢量。
\begin{figure}[H]
  \centering
  \includegraphics[width=.8\textwidth]{pict/PC/Inter/6.png} %1.png是图片文件的相对路径
  \caption{匹配示意图} %caption是图片的标题
  \label{PC:匹配示意图} %此处的label相PC:当于一个图片的专属标志,   目的是方便上下文的引用
\end{figure}
常用的匹配准则有:
\begin{enumerate}
  \item 均方误差(MSE)最小准则:	$$MSE(i, j)=\frac{1}{MN}\sum_{x=1}^M\sum_{y=1}^N[f_t(x, y)-f_{t-1}(x+i, y+j)]^2$$
  \item 平均绝对值误差(MAD)最小准则:$$MAD(i, j)=\frac{1}{MN}\sum_{x=1}^M\sum_{y=1}^N|f_t(x, y)-f_{t-1}(x+i, y+j)|$$
  \item 归一化相关函数(NCFF)最小准则:$$NCFF(i, j)=\frac{\sum_{x=1}^M\sum_{y=1}^Nf_t(x, y)f_{t-1}(x+i, y+j)}{\sqrt{\sum_{x=1}^M\sum_{y=1}^Nf^2_t(x, y)}\sqrt{\sum_{x=1}^M\sum_{y=1}^Nf^2_{t-1}(x+i, y+j)}}$$
  \item 绝对值误差(SAD)最小准则:$$SAD(i, j)=\sum_{x=1}^M\sum_{y=1}^N|f_t(x, y)-f_{t-1}(x+i, y+j)|$$
\end{enumerate}

\subsubsection{运动搜索算法介绍}%2.2.4

\paragraph{全局搜索算法}%2.2.4.1
为当前帧的一个给定块确定最优位移矢量的全局搜索算法方法是:在一个预先定义的搜索区域内,   把它与参考帧中所有的候选块进行比较,   并且寻找具有最小匹配误差的一个。这两个块之间的位移就是所估计的MV,   这样做带来的结果必然导致极大的计算量。

\paragraph{菱形搜索法}%2.2.4.2
对于视频的搜索点寻找方面, 鉴于全搜索算法的简单粗暴式搜索和三步搜索法的简单减少点数的搜索,   都不能很好的均衡搜索点和搜索效率的问题。由于绝大多数视频时慢速或者平缓运动, 所以一般的运动矢量分布范围都在以零运动矢量为中心的一个半径圆内, 同时不同形状和大小的搜索路径对整个算法的有效性和速度也会产生较大的影响。作为影响块匹配性能的主要点就是规定搜索模板的形状和大小。在菱形模型中,   为了解决局部最优和搜索精度问题,   设计了两种搜索模板:大菱形的搜索模板(LDSP)以及小菱形的搜索模板(SDSP)。 具体搜索点如\ref{PC:菱形搜索法搜索点示意图}:
\begin{figure}[H]
  \centering
  \includegraphics[width=.8\textwidth]{pict/PC/Inter/7.png} %1.png是图片文件的相对路径
  \caption{菱形搜索法搜索点示意图} %caption是图片的标题
  \label{PC:菱形搜索法搜索点示意图} %此处的label相PC:当于一个图片的专属标志,   目的是方便上下文的引用
\end{figure}
利用大钻石模型(LDSP)进行粗定位,   避免小模型进入局部最优,   随后在确定了粗模型时,   利用小模型(SDSP) 模板进行精确定位,   由于搜索模板会存在一.些重叠点,   因此在搜索时,   搜索效率更高,   且由于搜索范围变大,   搜素的结果也更好,   经过学者大量的实验证明,   钻石模型算法对运动细微和背景静态的视频序列最佳,   同时在算法效率是也提高了很多,   因此在H.264和HEVC都有采用此模型作为搜索模型的一种方案。同期出现的六边形模型, 和钻石模型的主要区别也是搜索点数的位置变化,   利用的核心思想类似,   这里就不在赘述。
\begin{figure}[H]
  \centering
  \includegraphics[width=.8\textwidth]{pict/PC/Inter/8.png} %1.png是图片文件的相对路径
  \caption{大钻石模型和小钻石模型} %caption是图片的标题
  \label{PC:大钻石模型和小钻石模型} %此处的label相PC:当于一个图片的专属标志,   目的是方便上下文的引用
\end{figure}


\paragraph{TZSearch算法}%2.2.4.3
\begin{enumerate}
  \item 确定起始搜索点。HEVC中采用高级运动向量预测技术(AMVP)技术来确定起始搜索点,   AMVP会给出若干个候选预测MV,   编码器从中选择率失真代价最小的作为预测MV,   并用其所指向的位置作为起始搜索点。
  \item 以步长1开始,   按下图所示的菱形模板在搜索范围内进行搜索,   其中步长以2的整数次幂的形式递增,   选出率失真代价最小的点作为该步骤的搜索结果。
  \item 若步骤2选出的最优点对应的步长为1,   则需要在该点周围进行二点搜索,   目的是补充搜索最优点周围尚未搜索的点。例如,   如果0上侧的1是最优点,   则需要对图中的搜索两个黑块。
  \item 若步骤2选出的最优点对应的步长大于某个阈值,   则以该最优点为中心,   在一定范围内做全搜索(搜索该范围内的所有点),   选择率失真代价最小的作为最优点。
  \item 以步骤4得到的最优点为起始点,   重复步骤2-4,   细化搜索。当相邻两次细化搜索得到的最优点一致时停止细化搜索。此时得到的MV即为最终MV。
\end{enumerate}

\begin{figure}[H]
  \centering
  \includegraphics[width=.8\textwidth]{pict/PC/Inter/9.png} %1.png是图片文件的相对路径
  \caption{TZSearch算法示意图} %caption是图片的标题
  \label{PC:TZSearch算法示意图} %此处的label相PC:当于一个图片的专属标志,   目的是方便上下文的引用
\end{figure}

\subsubsection{MV预测技术}%2.2.5
空域上相邻块的MV具有较强的相关性;同时,   MV在时域上也具有一定的相关性。若利用空域或者时域上相邻块的MV对当前块的MV进行预测,   仅对预测残差进行编码,   则能够大量节省MV的编码比特数。H.265/HEVC在MV的预测方面提出了两种新的技术--Merge技术和AMVP技术。--Merge和AMVP技术都使用了空域和时域MV预测的思想,   通过建立候选MV列表,   选取性能最优的一个作为当前PU的预测MV。

\paragraph{Merge模式}
Merge模式会为当前PU建立一个MV候选列表,   列表中存在5个候选MV(及其对应的参考图像)。通过遍历5个候选MV,   并进行率失真代价计算,   最终选取率失真代价最小的一个作为该Merge模式的最优MV。若编/解码端依照相同的方式建立该候选表,   则编码器只需要传输最优MV在候选列表中索引即可,   这样大幅节省了运动信息的编码比特数。\\
\subparagraph{空域候选列表的建立}

空域最多提供4个候选MV,   即最多使用图中5个候选块中的4个候选块的运动信息,   列表按照A1-B1-B0-A0-(B2)的顺序建立,   其中B2为替补,   当A1,   B1,   B0,   A0中只有一个或多个不存在时,   需要使用B2的运动信息。
 \begin{figure}[H]
  \centering
  \includegraphics[width=.8\textwidth]{pict/PC/Inter/10.png} %1.png是图片文件的相对路径
  \caption{} %caption是图片的标题
  \label{PC:} %此处的label相PC:当于一个图片的专属标志,   目的是方便上下文的引用
\end{figure}
\subparagraph{时域候选列表的建立}

利用当前PU在邻近已编码图像中对应位置PU(同位PU:和当前帧中当前块处于相同的空间位置)的运动信息。与空域情况不同,   时域候选列表不能直接使用候选块的运动信息,   而需要根据参考图像的位置关系做相应的比例伸缩调整。cur\_PU表示当前PU,   col\_PU表示同位PU,   td和tb分别表示当前图像cur\_pic、同位图像col\_pic与二者参考图像cur\_ref、PC:col\_ref之PC:间的距离,   则当前PU的时域候选MV为:$curMV = \frac{td}{tb}colMV$
 \begin{figure}[H]
  \centering
  \includegraphics[width=.8\textwidth]{pict/PC/Inter/11.png} %1.png是图片文件的相对路径
  \caption{} %caption是图片的标题
  \label{PC:} %此处的label相PC:当于一个图片的专属标志,   目的是方便上下文的引用
\end{figure}


\paragraph{AMVP模式(Inter模式)}
高级运动向量预测(Advanced Motion Vector Prediction,  AMVP)利用空余、时域上运动向量的相关性,   为当前PU建立了候选预测MV列表。编码器从中选出最优的预测MV,   并对MV进行差分编码;解码端通过建立相同的列表,   仅需要运动向量残差(MVD)与预测MV在该列表中的序号即可计算出当前PU的MV。

类似于Merge模式,   AMVP候选MV列表也包含空域和时域两种情形,   不同的是AMVP列表长度仅为2。
 \begin{figure}[H]
  \centering
  \includegraphics[width=.8\textwidth]{pict/PC/Inter/12.png} %1.png是图片文件的相对路径
  \caption{} %caption是图片的标题
  \label{PC:} %此处的label相PC:当于一个图片的专属标志,   目的是方便上下文的引用
\end{figure}
AMVP候选列表构建流程中空域的5个位置和merge模式下空域的5个位置完全相同,   但最终选择的是两个最优位置,   其中一个来自上边块,   另一个来自左边块。而时域运动矢量的选取是利用两个不同预测方向的时域相邻预测单元的运动矢量作为测量值,   并选取最优的一个作为时域运动矢量。当时域和空域候选子集选取完成后,   首先去除重复的运动矢量,   其次检查运动矢量的总数是否为2,   若大于2则保留前两个即去除索引值大于1的,   若小于2则添加零运动矢量。

\subsection{结语}%2.3
可见,   帧间预测是指利用视频时间域的相关性,   使用邻近已编码图像像素预测当前图像的像素,   以达到有效去除视频时域冗余的目的。由于视频序列通常包括较强的时域相关性,   因此预测残差通常是“平坦的”,   即很多残差值接近于“0”。将残差信号作为后续模块的输入进行变换、量化、扫描及熵编码,   可实现对视频信号的高效压缩。
      \section{变换编码与量化}
        \subsection{变换编码}
\subsubsection{变换编码的起因和基本概念}
就经验而言,   绝大多数图像都有一个共同特征:平坦和缓变区域占据大部分,   而细节和突变区域则占小部分。这样,   空间域的图像变换到频域或其它变换域,   会产生相关性很小的一些变换系数,   并可对其进行压缩编码,   即所谓的变换编码。

变换中有一类叫做正交变换,   可用于图像编码。自1968年利用快速傅立叶变换(FFT)进行图像编码以来,   出现了多种正交变换编码方法,   如 K-L变换(Karhunen-Loève Transform)、离散余弦变换(Discrete Cosine Transform,   DCT)等。其中,   编码性能以 K-L 变换最理想,   但缺乏快速算法,   且变换矩阵随图像而异。当信号具有接近马尔可夫过程的统计特性时,   离散余弦变换的去相关性接近于K-L变换的性能,   并且具有快速算法,   因此广泛应用于图像视频编码。

\subsubsection{K-L变换编码及其实现}
出于压缩的需要,   我们希望对于输入进行一个变换,   使得变换后的输出特征数较之前更少,   且各特征之间几乎正交,   在考虑均方误差最小的情况下,   就产生了K-L变换。

将图像阵列用向量形式表示,   也即$x=(x_0, x_1, \cdots, x_{N-1})^T$,   我们的目标是找到一种变换矩阵U,   使得输出:$y=U^Tx$。并且由于输出向量Y各分量之间是相互独立的,   因此有$E(y_iy_j)=0, \forall i\neq j$成立。若令分别为向量$x, y$的自相关矩阵$R_x=E(xx^T), R_y=E(yy^T)$。

考虑到的特征向量正交,   所以如果将U的列向量取为$R_x$的行向量,   这时$R_y=diag(\lambda_1, \cdots, \lambda_N)$, $\lambda_1, \cdots, \lambda_N$为$R_x$特征值。这说明经过K-L变换后,   消除了列向量Y的相关性,   能量只集中在这从大到小排列的N个特征值$\lambda_i$上,   因此编码只需要传送这N个特征值,   就可以大大降低码率。

对于在均方误差最小准则下,   K-L变换是失真最小的变换,   是一种理想的变换。

\subsubsection{K-L变换效果分析}
由于矩阵运算在matlab等数学软件上较为简单快捷,   故采用matlab作为实验软件,   采用K-L变换对20.9KB的401*150PNG灰度图像进行K-L变换压缩,   分别进行压缩率为48.44\%、73.44\%和85.94\%的图像压缩。实验结果如图\ref{TQ:KL}。
\begin{figure}
  \centering
  \subfloat[原图]{\includegraphics{pict/TQ/K-Luncompressed.png}}\\
  \subfloat[48.44\%压缩率]{\includegraphics{pict/TQ/K-Ltest_result1.png}}\\
  \subfloat[73.44\%压缩率]{\includegraphics{pict/TQ/K-Ltest_result2.png}}\\
  \subfloat[85.94\%压缩率]{\includegraphics{pict/TQ/K-Ltest_result3.png}}
  \caption{K-L变换效果图}
  \label{TQ:KL}
\end{figure}
可以看到,   随着压缩率的提升,   图像质量也逐渐下降。并且K-L在实际运用中没有较为便利的算法,   且时间消耗消耗大。在实际变换编码中主要还是采用DCT变换编码为主。

\subsubsection{DCT变换编码及其实现}
傅里叶变换表明,   任何信号能够便是为多个不同振幅和频率的正弦波或余弦信号的叠加。如果采用的是原函数,   且输入是离散的,   则称之为离散余弦变换(Discrete Cosine Transform,   DCT)。采用DCT可以使信号能量集中于少数几个系数中,   便于信源压缩,   如量化、熵编码等。

数学上共存在8种类型的DCT,   在图像处理中常用形式如下,   对应偶数阶的实偶DFT:
\begin{align}
  X(k)&=\sqrt{\frac{2}{N}}\epsilon_k\sum_{n=0}^{N-1}x(n)\cos{\left[\frac{(2n+1)k\pi}{2N}\right]}, k=0, 1\cdots, N-1\label{TQ:typeI}\\
  X(k)&=\sqrt{\frac{2}{N}}\epsilon_n\sum_{n=0}^{N-1}x(n)\cos{\left[\frac{(2k+1)n\pi}{2N}\right]}, k=0, 1\cdots, N-1\label{TQ:typeII}\\
  \epsilon_p&=\begin{cases}
    \frac{1}{\sqrt{2}},  \qquad p=0\\
    1,  \qquad otherwise
  \end{cases}
\end{align}
其中\ref{TQ:typeI}对应一维正变换,   \ref{TQ:typeII}对应一维反变换。

图像、视频编码主要使用二维DCT, 形式如下:
\begin{equation}
  \begin{array}{c}
    X(k, l)=C(k)C(l)\sum_{m=0}^{N-1}\sum_{n=0}^{N-1}x(m, n)\cos{\left[\frac{(2m+1)k\pi}{2N}\right]}\cos{\left[\frac{(2n+1)l\pi}{2N}\right]}\\
    k, l=0, 1\cdots, N-1
  \end{array}
\end{equation}
其中
\begin{equation}
  C(k)=C(l)=\begin{cases}
    \sqrt{\frac{1}{N}}, \qquad k, l=0\\
    \sqrt{\frac{2}{N}}, \qquad otherwise
  \end{cases}
\end{equation}
其逆变换如下:
\begin{equation}
  \begin{array}{c}
    X(m, n)=\sum_{m=0}^{N-1}\sum_{n=0}^{N-1}C(k)C(l)x(k, l)\cos{\left[\frac{(2m+1)k\pi}{2N}\right]}\cos{\left[\frac{(2n+1)l\pi}{2N}\right]}\\
    m, n=0, 1\cdots, N-1
  \end{array}
\end{equation}

二维DCT可视为对列进行一次DCT后再对行进行一次DCT, 以矩阵形式表达为$Y=AXA^T$, 其中A为一维DCT变换矩阵, 反变换为$X=A^TYA$。

如图\ref{TQ:JPEG}为二维$8\times 8$DCT基图像,   二维DCT将图像变换为64个块的加权和,   其中权重即为DCT系数。
\begin{figure}
  \centering
  \includegraphics[width=.5\textwidth]{pict/TQ/DCT-8x8.png}
  \caption{Two-dimensional DCT frequencies from the \href{https://en.wikipedia.org/wiki/JPEG\#Discrete_cosine_transform}{JPEG DCT}}
  \label{TQ:JPEG}
\end{figure}
对于灰度值缓慢变化的像素块来说, 经过DCT后绝大部分能量都集中在左上角的低频系数中;相反,   如果像素块包含较多细节纹理信息, 则较多能量分散在高频区域。实际上大多数图像包含更多的低频分量,   并且人眼对高频细节相对不敏感, 因此可以对高能量的低频系数进行较为精细的量化与处理, 这样可以较好地压缩图像而不会造成明显的主观质量下降。

如图\ref{TQ:astronaut}, 可见该图像大多数变换系数都集中在0附近, 且主要分布在低频区域。
\begin{figure}
  \centering
  \subfloat[原图]{\includegraphics[width=.4\textwidth]{pict/TQ/ori.png}}\\
  \subfloat[变换系数空间分布]{\includegraphics[width=.8\textwidth]{pict/TQ/dist2.png}}
  \caption{$512\times 512$图像astronaut DCT系数分布图}
  \label{TQ:astronaut}
\end{figure}
图\ref{TQ:rec}表示了使用部分大系数对该图像重建结果:
\begin{figure}
  \centering
  \subfloat[原图]{\includegraphics[width=.45\textwidth]{pict/TQ/astronaut.png}}\hspace*{\fill}\subfloat[51.5\%系数重建结果, PSNR=42.63dB]{\includegraphics[width=.45\textwidth]{pict/TQ/rec_42_63_515.png}}\\
  \subfloat[28.0\%系数重建结果, PSNR=36.25dB]{\includegraphics[width=.45\textwidth]{pict/TQ/rec_36_25_280.png}}\hspace*{\fill}\subfloat[5.27\%系数重建结果, PSNR=28.10dB]{\includegraphics[width=.45\textwidth]{pict/TQ/rec_28_10_0527.png}}
  \caption{$512\times 512$图像astronaut重建结果}
  \label{TQ:rec}
\end{figure}
可见, 仅采用部分变换系数便可很好地还原原图像, 因此使用DCT变换可以对图像进行明显的压缩。
\subsubsection{整数DCT}
在实际应用DCT时由于计算机精度有限不可避免地会带来舍入误差以及编/解码端正反变换失配的问题。针对这个问题, H.264/AVC与H.265/HEVC均采用了整数DCT变换来解决舍入误差以及编解码失配问题, 同时整数的使用提高了DCT的运算速度。

这里我们并不讨论HEVC中整数DCT的设计准则, 具体设计准则可以参考\cite{hevc}, 仅给出结果如图\ref{TQ:mat}。
\begin{figure}
  \centering
  \includegraphics[width=.8\textwidth]{pict/TQ/mat.png}
  \caption{$32\times 32$变换矩阵左半元素, 阴影处分别代表$16\times 16, 8\times 8, 4\times 4$变换矩阵矩阵}
  \label{TQ:mat}
\end{figure}
实际应用中在正逆变换处有对应的缩放因子, 并且要与量化结合使用。

在HEVC中帧内预测$4\times 4$模式亮度残差编码中使用$4\times 4$整数DST(Discrete Sine Transform), 式\ref{TQ:DST}给出了最终变换矩阵。
\begin{equation}
  H=\left[
  \begin{array}{cccc}
    29 & 55 & 74 & 84\\
    74 & 74 & 0 & -74\\
    84 & -29 & -74 & 55\\
    55 & -84 & 74 & -29
  \end{array}
  \right]
  \label{TQ:DST}
\end{equation}

\subsection{量化}
量化(Qualification)是指将信号的连续取值(大量可能的离散取值)映射为有限个离散幅值的过程,   实现多对一的映射。在视频编码中,   残差信号经过离散余弦变换(DCT)后,   通过量化变换系数可以有效地压缩信息。然而量化不可避免地会引入失真,   这也是视频编码中产生失真的根本原因,   因此它是视频编码中一个重要的环节。

\subsubsection{标量量化}
一个量化器可以由其输入端的范围划分方式以及对应的输出值唯一确定。根据输入输出数据的类型,   分为标量量化器和矢量量化器。其中标量量化器因其复杂度低、易实现的特性而广泛应用在各种图像、视频编码中。

一般标量量化器的原理是将一个幅度连续的信号映射成若干个离散的信号,   而这样做不可避免地会带来量化失真。主流衡量量化失真主要有三种准则:均方误差(MSE)、信噪比(SNR)和峰值信噪比(PSNR),   计算公式分别为:
\begin{align}
  MSE & = \frac{1}{M}\sum_{k=1}^M(x_k-\hat{x}^k)^2\\
  SNR & = 10lg\frac{\sigma_x^2}{MSE}\\
  PSNR & = 10lg\frac{x_{peak}^2}{MSE}
\end{align}

在图像和视频编码中,   常利用Lloyd-Max量化器和熵编码量化器进行量化,   并且认为DCT系数服从0均值的Laplace分布(其0值附近的系数概率较大),   因此常常加宽0值附近的区间,   从而带来率失真性能的提高。

\subsubsection{量化在H.265/HEVC中的应用}
\paragraph{量化}
H.265/HEVC标准仅规定了反量化过程的实现方法,   而将量化方法留给编码器决定,   这就留下了充足的优化量化方法的余地,   例如自适应量化(Adaptive Qualification)和率失真优化量化(RDOQ)等。
传统标量量化方法可以表示如下:
\begin{equation}
  l_i=floor(\frac{c_i}{Q_{step}}+f)
\end{equation}
其中,   $c_i$表示DCT系数,   $Q_{step}$表示量化步长,   $l_i$为量化后的值,   $floor$为向下取整函数,   而通过$f$来控制舍入关系。H.265/HEVC标准规定了52个量化步长,   对应于相同数量的量化参数(Qualification Parameter,   QP)(0--51)。两者关系近似由下式给出:
\begin{equation}
  Q_{step}\approx 2^{(QP-4)/6}
\end{equation}

从上式可以看出,   QP每增加6,   $Q_{step}$大约增大1倍。因此量化步长的变化范围相当宽,   可以根据不同的需求灵活选择。

需要注意的是,   对于色差信号若使用较大的量化步长会出现颜色漂移现象。为了应对这一问题,   H.265/HEVC标准将色差信号的量化参数限制在跟小的范围内(0--45),   具体对应关系见表\ref{TQ:Qstep}。
\begin{table}
  \caption{QP--$Q_{step}$对应关系}
  \label{TQ:Qstep}
  \begin{center}
    \begin{tabular}{|c|*{9}{c|}}\hline
      QP & $Q_{step}$ & QP & $Q_{step}$ & QP & $Q_{step}$ & QP & $Q_{step}$ & QP & $Q_{step}$\\\hline
      0 & 0.625 & 11 & 2.25 & 22 & 8 & 33 & 28.5 & 44 & 102\\\hline
      1 & 0.7031 & 12 & 2.5 & 23 & 9 & 34 & 32 & 45 & 114\\\hline
      2 & 0.7969 & 13 & 2.8125 & 24 & 10 & 35 & 36 & 46 & 128\\\hline
      3 & 0.8906 & 14 & 3.1875 & 25 & 11.25 & 36 & 40 & 47 & 144\\\hline
      4 & 1 & 15 & 3.5625 & 26 & 12.75 & 37 & 45 & 48 & 160\\\hline
      5 & 1.125 & 16 & 4 & 27 & 14.25 & 38 & 51 & 49 & 180\\\hline
      6 & 1.25 & 17 & 4.5 & 2.8 & 16 & 39 & 57 & 50 & 204\\\hline
      7 & 1.4062 & 18 & 5 & 29 & 18 & 40 & 64 & 51 & 228\\\hline
      8 & 1.5938 & 19 & 5.625 & 30 & 20 & 41 & 72 & & \\\hline
      9 & 1.7812 & 20 & 6.375 & 31 & 22.5 & 42 & 80 & & \\\hline
      10 & 2 & 21 & 7.125 & 32 & 25.5 & 43 & 90 & &\\\hline
    \end{tabular}
  \end{center}
\end{table}

H.265/HEVC的量化过程还将完成整数DCT中的比例缩放运算,   并且为了避免浮点数的运算,   量化器通过先放大后取整的方法进行变换。考虑到QP和之间的关系,   可以将QP表示为:
\begin{equation}
  QP=floor(QP/6)+QP\%6
\end{equation}
再引入变量qbits和MF:
\begin{align}
  qbits &= 14 + floor(QP/6)\\
  MF &= \frac{2^{qbits}}{Q_{step}}\begin{cases}
    26214 \qquad QP\%6=0\\
    23302 \qquad QP\%6=1\\
    20560 \qquad QP\%6=2\\
    18396 \qquad QP\%6=3\\
    16384 \qquad QP\%6=4\\
    14564 \qquad QP\%6=5
  \end{cases}
\end{align}

其中\%表示取余运算。又考虑到整数DCT中的缩放因子通常为2的整数次幂$2^{T_{shift}}$,   则上式可以写为:
\begin{equation}
  l_{ij}=floor(\frac{d_{ij}MF}{2^{qbits+T_{shift}}}+f)=(d_{ij}MF+f')>>(qbits+T_{shift})
\end{equation}
其中,   $>>$表示右移运算,   $d_{ij}$表示缩放前的DCT系数,   舍入偏移则由$f'=f<<(qbits+T_{shift})$代表。

综上所述,   H.265/HEVC的量化公式为:
\begin{align}
  |l_{ij}|&=(|d_{ij}|MF+f')>>(qbits+T_{shift})\\
  sign(l_{ij})&=sign(d_{ij})
\end{align}

\paragraph{反量化}
对应于上述的量化公式,   容易得出标量量化的反量化公式为:
\begin{equation}
  \hat{c}_i=l_i Q_{step}
\end{equation}
其中$\hat{c}_i$表示反量化后得到的重构DCT系数。由于量化是一个有损过程,   因此通常情况下$\hat{c}_i\neq c_i$。

和量化类似的是,   反量化过程也全部使用整数进行,   同时亦融合了IDCT中的比例伸缩运算。引入变量shift和scale如下所示:
\begin{align}
  shift &= 6 + floor(QP/6)-IT_{shift}\\
  scale &=26\cdot Q_{step}=2floor(QP/6)\cdot\begin{cases}
    40 \qquad QP\%6=0\\
    45 \qquad QP\%6=1\\
    51 \qquad QP\%6=2\\
    57 \qquad QP\%6=3\\
    64 \qquad QP\%6=4\\
    72 \qquad QP\%6=5\\
  \end{cases}
\end{align}
其中$IT_{shift}$表示IDCT中的比例伸缩。综上,   H.265/HEVC的反量化公式为:
\begin{equation}
  \hat{c}_{ij}=(l_{ij}\cdot scale+(1<<(shift-1)))>>shift
\end{equation}

\paragraph{率失真优化量化(RDOQ)}
传统的标量量化器是以失真最小为目的进行设计的,   而正如前文所述,   在视频编码中,   编码比特率和失真是需要权衡的。率失真优化量化(RDOQ)就是这样一种量化器。其主要思想为将量化过程同率失真优化相结合,   在多个可选量化值中选择一个最优的:
\begin{equation}
  l_i^*=\arg\min_{k=1, \cdots, m}\{D(c_i, l_{i, k})+\lambda\cdot R(l_{i, k})\}
\end{equation}
其中,   $D(c_i, l_{i, k})$为$c_i$量化为$l_{i, k}$时的失真,   $R(l_{i, k})$表示$c_i$量化为$l_{i, k}$时的编码比特数,   $\lambda$为拉格朗日因子,   $l_i^*$即为最优量化值。

依照上述原则,   在H.265/HEVC中实现方法具体如下:
\begin{enumerate}
  \item 确定当前TU每个系数的可选量化值。用下式对当前TU所有系数进行预量化$|l_i|=round(\frac{|c_i|}{Q_{step}})$
  \item 利用RDO准则确定当前TU所有系数的最优量化值。按扫描顺序遍历当前TU所有系数,   对于每一个系数,   遍历其可选量化值,   并利用RDO确定最优量化值。
  \item 用RDO准则确定当前TU所有系数块组(CG)是否优化为全零组。由于在熵编码(CABAC)的过程中,   全零CG只需要编码全零标识,   省去了许多步骤,   如果当前CG仅含有极少个数且幅值较小的系数时,   优化为全零CG可能会获得更好的率失真性能。
  \item 利用RDO准则确定当前TU“最后一个非零系数”的位置,   这样可以省去CABAC中TU编码拖尾零系数的比特数。因此其位置对失真和编码比特数有着严重的影响。
\end{enumerate}

和标量量化相比,   RDOQ提高了编码器的性能,   但由于需要遍历多个可选量化值并计算率失真代价,   其编码复杂度也有一定增加。实验结果表明,   RDOQ能使编码性能提高3\%--6\%,   但总编码时间大约增加10\%--15\%。

\paragraph{量化参数}
在视频编码中,   QP是相当重要的参数,   直接影响着视频的编码比特率。对于某些应用场合,   尤其是传输速率受限时,   灵活地控制量化参数使得编码速率满足要求显得尤为重要。为此H.265/HEVC制定了十分灵活的QP控制机制——量化组(Qualification Group,   QG)。规定一个CTB可以拥有一个或多个固定的QG,   同一个QG内的所有含有非零系数的CU共享一个QP,   而不同的QG可以使用不同的QP。这样便通过增加QP解析算法的复杂度来进行更灵活的速率控制。

\subparagraph{QG概念}
QG是指将一幅图像分成固定大小的$N\times N$的正方形像素块,   具体由图像参数集指定,   其必须处于最大CU和最小CU之间(同时包含两者),   图\ref{TQ:DCT4}给出了一个$32\times 32$的QG示意图。从图中可以明白,   CU和QG并无固定大小关系,   因为QG为固定大小,   而CU是根据视频内容自适应划分的。
\begin{figure}
  \centering
  \includegraphics[width=.4\textwidth]{pict/TQ/DCT4.png}
  \caption{QG边界与CU划分}
  \label{TQ:DCT4}
\end{figure}

\subparagraph{QP的预测编码}
H.265/HEVC进一步发展了H.264/AVC中对量化参数QP进行预测编码的思想,   它使用相邻已编码QG的信息来预测当前QG的QP,   增加了QP预测的准确度。下图\ref{TQ:DCT5}给出了H.265/HEVC中QP的预测示例,   其中A和B分别为当前QG左侧和上方的已编码QG,   则当前QG的预测QP应为:
\begin{equation}
  predQP=(QP_A+QP_B+1)>>1
\end{equation}
需要注意的是,   在实际QP预测过程中,   在某些情况下A和B可能不存在,   这时就应根据解码器的设置来将A和B进行替换。
\begin{figure}
  \centering
  \includegraphics[width=.3\textwidth]{pict/TQ/DCT5.png}
  \caption{QP的预测模板}
  \label{TQ:DCT5}
\end{figure}

\subparagraph{CU层QP的解析}
由于H.265/HEVC是以CU为单元进行解码,   因此QP的解析也依赖于CU。考虑到色度分量QP的计算使用了亮度的QP,   因此下文以CU亮度QP为例展示QP的解析过程。

由于H.265/HEVC对QP进行了预测编码,   因此QP的解析也即QP的预测值(predQP)和预测误差(deltaQP)的解析。

对于预测QP的获取,   一般分情况考虑。由于CU和QG没有固定的大小关系,   当一个QG包含一个及以上CU时,   所有CU使用同一个预测QP,   也就是当前QG的预测QP。反之,   则采用CU内的第一个预测QP作为当前CU的预测QP。

deltaQP的获取更为复杂,   但情况大致相同。当一个QG包含一个及以上CU时,   deltaQP会在解码顺序上的第一个含有非零系数的CU中传递。当前QG内在此之前所有CU的deltaQP都为0,   之后的所有CU都使用同一个QG。

反之,   则采用第一个含有非零系数的QG所携带的deltaQP。若所有的QG系数均为0,   那么该CU的deltaQP亦为0。

\subsubsection{量化矩阵}
H.265/HEVC使用了量化矩阵,   其原理是对不同的系数采用不同的量化步长。例如,   可以利用人眼对图像视频中高频细节不敏感的特征,   对高频系数采用较大的量化步长,   而对低频系数采用较小的量化步长,   这样做能够保证在保证一定压缩率的同时提高图像或视频的主观质量。

H.265/HEVC的变换量化过程依次为残差变换、比例缩放以及量化。其中量化矩阵主要作用于比例缩放部分,   其大小和TU相同,   从$4\times 4$到$32\times 32$不等,   变换后的DCT系数将和量化矩阵中对应位置的系数相除,   所得结果将作为量化模块的输入。

H.265/HEVC标准允许采用默认量化矩阵或用户自定义量化矩阵。对于默认量化矩阵,   由$4\times 4$和$8\times 8$两种大小,   更高阶的量化矩阵由$8\times 8$量化矩阵通过上采样得到。在默认量化矩阵中,   越靠近右下角的元素值一般越大,   说明高频系数会使用较大的量化步长,   这样能够符合人眼的特性,   提高编码视频的主观质量。

对于自定义矩阵,   H.265/HEVC允许编码器根据不同的应用场合自行决定量化矩阵各元素的值。此时,   量化矩阵需要被写入码流传送到解码器。为了节省资源,   一般使用差分编码完成。
      \section{环路滤波}
        通常在进行了量化和变换编码的过程之后,   原有视频会产生一些失真:例如方块效应、振铃效应、颜色偏差以及图像模糊等。为了降低失真的影响,   提高处理后的视频质量,   一般通过环路后处理修正。本文主要介绍并实现用以消除方块效应的“去方块滤波”和改善振铃效应的“样点自适应补偿”模块。两个模块在编码环路的位置如图所示,   可以看到,   去方块滤波模块和样点自适应补偿模块都处于编码环路中。也就是说经过滤波后的重构像素才能作为后续编码使用。而经过环路滤波处理后的重建像素更有利于参考,   进一步减小后续编码像素的预测残差,   有效提高图像质量。图\ref{LF:pic1}以框图形式表示了整个去方块滤波流程。
\begin{figure}[H]
  \centering
  \includegraphics{pict/LF/DEBLOCK0.PNG}
  \label{LF:pic1}
\end{figure}
\subsection{去方块滤波}
\subsubsection{方块效应和去方块滤波原理}
在编解码器反变换量化后图像会出现方块效应。其产生的原因有两个。最重要的一个原因是基于块的帧内和帧间预测残差的 DCT 变换。变换系数的量化过程相对粗糙,   因而反量化过程恢复的变换系数带有误差,   会造成在图像块边界上的视觉不连续。第二个原因来自于运动补偿预测。运动补偿块可能是从不是同一帧的不同位置上的内插样点数据复制而来。因为运动补偿块的匹配不可能是绝对准确的,   所以就会在复制块的边界上产生数据不连续。

尽管采用较小的变换尺寸可以降低这种不连续现象,   但仍需要一个去方块滤波器以最大程度提高编码性能。

去方块滤波器的作用是去除编解码算法带来的方块效应。但是,   如果在DCT边界上,   正好是图像的边界,   若不加以判断而误认为是方块效应,   则可能造成新的误差。为此,   在滤波方块效应时,   应该先判断该边界是图像真实边界还是方块效应所形成的边界(假边界)。对真实边界不进行滤波处理,   而对假边界则要根据周围图像块的性质和编码方法采用不同强度的滤波。
\subsubsection{去方块滤波算法构成}
	\paragraph{边界分析}

自适应边界滤波:

当对残差用DCT变换进行编码时,   方块边界比内部的编码误差大。对这个现象的合理解释是内部点的重建是对周围点进行加权平均得到。而边界点所用到的加权平均点较少,   所以重建效果较差。该误差分布不均匀的导致需要方块边界滤波以提高图像客观质量。

边界强度(Bs)决定去方块滤波器选择滤波参数,   控制去除方块效应的程度,   它与边界的性质有关。一般根据Bs与相邻图像块的模式及编码条件的关系,   给Bs相应赋值。

在实际滤波算法中,   Bs决定对边界的滤波强度,   包括对两个主要滤波模式的选择。Bs值的下降趋势说明最强的方块效应主要来自于帧内预测模式及对预测残差编码,   而在较小程度上与图像的运动补偿有关。色度块边界滤波的Bs值不另外单独计算,   而是从相应亮度块边界的Bs值复制而来。

在帧场自适应宏块中,   条件相对复杂些,   因为相邻两图像块中的一个可能来自帧编码宏块或来自场编码宏块。滤波强度变化的原则不变。为了避免将图像过度模糊化,   对于来自场编码宏块的水平边界需要特别考虑以避免过强的滤波强度,   这是因为这种宏块的垂直滤波的空间扩展范围是其它情况的两倍。

自适应样点滤波(Sample Adaptive Offset, SAO):

在去方块滤波中,   非常重要的是要区分图像中的真实边界和由 DCT 变换系数量化而造成的假边界。为了保持图像的逼真度,   应该尽量滤除假边界以不致被看出的同时保持图像真实边界不被滤波。
为了区分这两种情况,   要分析每个需要被滤波的边界两边的样点值。本文为方便起见,   以4×4 变换块为例。定义两个相邻4×4块中一条直线上的样点为 $p_3, p_2, p_1, p_0, q_0, q_1, q_2, q_3$,   实际的图像边界在$p_0$和$q_0$之间,   如图\ref{LF:pic2}所示。
\begin{figure}[H]
  \centering
  \includegraphics[width=.6\textwidth]{pict/LF/DEBLOCK1.png}
  \caption{典型不需要滤波的图像边界}
  \label{LF:pic2}
\end{figure}
如上所述,   当Bs值为0时,   滤波器对边界不起作用。对于Bs值为非 0的边界,   为区分上述真假两种边界,   定义一对与量化有关的参数,   为 $\alpha$和$\beta$,   用来检查图像内容,   以决定每个样本点集是否要被滤波。只有下述三个条件同时满足,   直线上的样点才被滤波:
\begin{equation}
| p_0-q_0 | <\alpha(Index_A), | p_1-p_0 | <\beta(Index_B), | q_1-q_0 | <\beta(Index_B)
\end{equation}
 $\alpha$和$\beta$值根据边界两边的平均量化参数查表得到,    $\alpha$和$\beta$的查表指数根据下式计算:

\begin{equation}
Index_A =
\begin{cases}
0,   & if\qquad  QP+Offset_A\leq 0 \\
QP+Offset_A,  & if\qquad  0<QP+Offset_A<51 \\
51,   & if\qquad  QP+Offset_A \geq 51
\end{cases}
\end{equation}

\begin{equation}
Index_B =
\begin{cases}
0,   & if\qquad  QP+Offset_B\leq 0 \\
QP+Offset_B,  & if\qquad  0<QP+Offset_B<51 \\
51,   & if\qquad  QP+Offset_B\geq 51
\end{cases}
\end{equation}
其中,   0到51为QP的范围。$Offset_A$和$Offset_B$为在编码器中选择的偏移值,   以在片级上控制去方块滤波的性能。$\alpha$和$\beta$值满足下面近似经验关系:
\begin{equation}
	\alpha(x)=0.8(2^{6/x}-1), \beta(x)=0.5x-7
\end{equation}
这个关系式中的变量是根据测试进行选择的,   让不同的内容得到满意的视觉效果。一般来说$\beta$比$\alpha$小。为了节省计算量,   $\alpha$和$\beta$值通过查表得到。特别地,   在表格低端,   两者取值被限为0,   这样对$Index_A$<16或$Index_B$<16,   $\alpha$和$\beta$中的一个或两个全部为0,   相应地不进行滤波。

\begin{table}[!htbp]
\begin{tabular}{|c|c|c|c|c|c|c|c|c|c|c|c|c|c|c|c|c|c|c|c|c|}
\hline
\multicolumn{21}{|c|}{$Index_A$(对应$\alpha$)或者$Index_B$(对应$\beta)$}\\ % 用\multicolumn{3}表示横向合并三列 
                        % |c|表示居中并且单元格两侧添加竖线 最后是文本
\hline
 &<16&16&17&18&19&20&21&22&23&24&25&26&27&28&29&30&31&32&33&34\\
\hline

$\alpha$&0&4&4&5&6&7&8&9&10&12&13&15&17&20&22&25&28&32&36&40\\
\hline
$\beta$&0&2&2&2&3&3&3&3&4&4&4&6&6&7&7&8&8&9&9&10\\
\hline
\end{tabular}
\begin{tabular}{|c|c|c|c|c|c|c|c|c|c|c|c|c|c|c|c|c|c|}
\hline
\multicolumn{18}{|c|}{$Index_A$(对应$\alpha$)或者$Index_B$(对应$\beta)$}\\ % 用\multicolumn{3}表示横向合并三列 
                        % |c|表示居中并且单元格两侧添加竖线 最后是文本
\hline
 &35&36&37&38&39&40&41&42&43&44&45&46&47&48&49&50&51\\
\hline

$\alpha$&45&50&56&63&71&80&90&101&113&127&144&162&182&203&226&255&255\\
\hline
$\beta$&10&11&11&12&12&13&13&14&14&15&15&16&16&17&17&18&18\\
\hline
\end{tabular}
\caption{关系}
\end{table}

$\alpha$和$\beta$与QP的关系将滤波强度与滤波前重建的图像一般质量联系起来。因为阈值随QP增加,   当QP较大时,   含有较多内容的边界需要被滤波,   这是由于编码误差随QP增加的缘故。$\alpha$中的指数特性反映期望的方块效应与$\alpha$的关系,   因为QP每增加6则量化步长增加一倍。

\paragraph{滤波过程}

基本滤波运算:

先讨论对亮度点的滤波。对这种模式的滤波,   滤波后的 和 值按下式计算:
\begin{equation}
p_0^{'} =p_0+\Delta_0, q_0^{'} =q_0-\Delta_0
\end{equation}

其中 $\Delta_0$分两步计算,   先计算它的初始值$\Delta_{0i}$ ,   再对这个初始值进行限幅后代入上式。初始值$\Delta_{0i}$ 根据边界两边的样点值计算:
\begin{equation}
\Delta_{0i}=(4(q_0-p_0)+(p_1-q_1)+4)>>3
\end{equation}
当上述式子成立时,   才修正对应$p_1$ 或$q_1$ 值,   即滤波后的$p_1$ 或$q_1$  值按下式计算:
\begin{equation}
p_1^{'} =p_1+\Delta_p, q_1^{'} =q_1-\Delta_q
\end{equation}
同时计算$p_1^{'} $的初始值$\Delta$为$\Delta_{p1i}=(p_2+((p_0+q_0+1)>>1)-2P_1)>>1, \Delta_{q1i}$按同样关系式计算,   用$q_2$ 和$q_1$ 分别代替$p_2$ 和$p_1$ 即可。可以证明,   上式的脉冲响应具有很强的低通特性。

限幅:

如果上述初始值$\Delta_{0i}$、$\Delta_{p1i}$和$\Delta_{q1i} $ 直接应用在滤波计算中,   则可能导致滤波频率过低,   出现图像模糊。自适应滤波器的一个重要部分是限制 的值,   称为限幅。对于内部和边界上的样点,   限幅过程不同。
\begin{table}[!htbp]
\begin{tabular}{|c|c|c|c|c|c|c|c|c|c|c|c|c|c|c|c|c|c|c|c|}
\hline
\multicolumn{20}{|c|}{$Index_A$}\\ 
\hline
 &<17&17&18&19&20&21&22&23&24&25&26&27&28&29&30&31&32&33&34\\
\hline
Bs=1&0&0&0&0&0&0&0&1&1&1&1&1&1&1&1&1&1&2&2\\
\hline
Bs=2&0&0&0&0&0&1&1&1&1&1&1&1&1&1&1&2&2&2&2\\
\hline
Bs=3&0&1&1&1&1&1&1&1&1&1&1&2&2&2&2&3&3&3&4\\
\hline
\end{tabular}
\begin{tabular}{|c|c|c|c|c|c|c|c|c|c|c|c|c|c|c|c|c|c|}
\hline
\multicolumn{18}{|c|}{$Index_A$}\\ 
\hline
 &35&36&37&38&39&40&41&42&43&44&45&46&47&48&49&50&51\\
\hline
Bs=1&2&2&3&3&3&4&4&4&5&6&6&7&8&9&10&11&13\\
\hline
Bs=2&3&3&3&4&4&5&5&6&7&8&8&10&11&12&13&15&17\\
\hline
Bs=3&4&4&5&6&6&7&8&9&10&11&13&14&16&18&20&23&25\\
\hline
\end{tabular}
\caption{滤波限幅变量c1值与$Index_A$和Bs的关系}
\end{table}
对于内部样点,   用于滤波的 值$\Delta$被限制在$-c_1$ 到$c_1$ 范围内,    是从二维表中查找的参数,   它是根据用于计算$\alpha$的IndexA和Bs查找。IndexA和Bs越大,   则 也越大,   也就允许更强的滤波。最终$p_1$和$q_1$滤波的限幅值为:
\begin{equation}
  \Delta_{p1}=\begin{cases}
    -c_1 \qquad \Delta_{p1i}<-c_1\\
    \Delta_{p1i} \qquad -c_1\leq\Delta_{p1i}\leq c_1\\
    c_1 \qquad \Delta_{p1i}>c_1
  \end{cases}\qquad
  \Delta_{q1}=\begin{cases}
    -c_1 \qquad \Delta_{q1i}<-c_1\\
    \Delta_{q1i} \qquad -c_1\leq\Delta_{q1i}\leq c_1\\
    c_1 \qquad \Delta_{q1i}>c_0
  \end{cases}
\end{equation}

对于滤波边界$p_0$ 和$q_0$ 样点,   $\Delta_{0i}$ 的限幅值由 $c_1$和条件决定。先将它的限幅值 $c_0$定为$c_1$ 。

如果滤波条件都成立,   说明边界两边内部的变化强度小于$\beta$阈值,   需要对边界进行更强的滤波(同时如上述需要对 $p_1$或$q_1$ 样点进行修正),   $c_0$ 将增加1。这样对边界样点的修正值为:
\begin{equation}
  \Delta_0=\begin{cases}
    -c_0 \qquad \Delta_{0i}<-c_0\\
    \Delta_{0i} \qquad -c_0\leq\Delta_{0i}\leq c_0\\
    c_0 \qquad \Delta_{0i}>c_0
  \end{cases}
\end{equation}

对色度点滤波,   只有 $p_0$和$q_0$ 才被修正。滤波方法与亮度点一样,   只是限幅值$c_0$ 为 $c_1$加1。这样对Bs小于4的边界没有必要对色度进行估计,   也不必存取变量$p_2$ 和$q_2$ 值。

平滑边界滤波:

H.265/MPEG-4 AVC的帧内编码在对同一图像区域编码时倾向采用 16×16亮度预测模式。这会在宏块边界引起小幅度的方块效应。但是由于Mach band效应,   在这种情况下,   即使是很小的强度值差别在视觉上的感觉是敏感的。为了消除这种效应,   需要在图像内容平滑的两个宏块边界采用较强的滤波器。

对亮度滤波,   根据图像内容判断选择较强的滤波器,   还是较弱滤波器。较强的滤波器对边界两边的边界点及两个内部点进行修正,   而较弱滤波器仅改变边界点。只有下面的跨边界差异的约束条件成立才使用较强的滤波器:

\begin{equation}
| p_0-q_0 |<(\alpha>>2)+2
\end{equation}
对于亮度滤波,   当满足一定条件时,   根据下式计算滤波后的样点值:
\begin{align}
  p_0^{'}&=(p_2+2p_1+2p_0+2q_0+2q_1+4)>>3\\
  p_1^{'}&=(p_2+p_1+p_0+q_0+2)>>2\\
  p_2^{'}&=(2p_3+3p_2+p_1+p_0+q_0+4)>>3
\end{align}
否则,   只根据下式修正$p_0$:
\begin{equation}
p_0^{'} =(2p_1+p_0+q_1+2)>>2
\end{equation}
q点值的修正方法类似,   但是在选择亮度滤波器时用相应的关系式代替。
\subsubsection{去方块滤波实现及其效果}
去方块滤波的具体实现因滤波顺序的不同存在多种方式:可以采用CTB为基本单位,   按照raster扫描方式进行处理;或者先将整幅图像划分为互不重叠的$8\times 8$块,   然后再进行滤波;或者以CU为基本单位,   按照Z扫描方式进行处理。但总体上都遵循对整幅图像先水平滤波再垂直滤波,   并且仅对$8\times 8$大小的块边界进行处理的原则。具体实现如下:
\paragraph{滤波顺序}
首先,   将图像大小划分为相同的CTB块,   按照raster扫描方式对每个CTB进行处理。其次,   对于每个CTB,   按照Z扫描方式以CU为基本单位进行处理,   如图所示,   设CTB大小为64$\times$64。如图\ref{LF:pic3}所示。
\begin{figure}[H]
  \centering
  \includegraphics[width=.5\textwidth]{pict/LF/DEBLOCK4.png}
  \caption{图像中CU的滤波顺序}
  \label{LF:pic3}
\end{figure}
每个亮度CU块对应2个色度块,   两者相互穿插进行滤波过程。
对垂直边界按照从左到右的顺序进行处理,   如图中第一个CU块,   它的亮度分量的滤波顺序为$a\to d$ ,   相应的,   其色度分量的滤波顺序为 $e\to f$,   $g\to h$,   如图\ref{LF:pic4}所示。
\begin{figure}[H]
  \centering
  \includegraphics[width=.5\textwidth]{pict/LF/DEBLOCK5.png}
  \caption{亮度/色度滤波顺序}
  \label{LF:pic4}
\end{figure}
\paragraph{具体过程}
去方块滤波过程具体共分为4个步骤,   其流程如图\ref{LF:pic5}所示
\begin{figure}[H]
  \centering
  \includegraphics{pict/LF/DEBLOCK6.png}
  \caption{去方块滤波流程}
  \label{LF:pic5}
\end{figure}
\subparagraph{确定滤波边界}
该部分的核心是确保被滤波的边界必然是PU或TU的边界,   并且图像边界不需要被滤波,   具体步骤如下:
\begin{enumerate}
\item 
将a, b, c, d的滤波标志均设为0。
\item
对边界a进行判断,   如果边界a不满足三种情况(a为图像的左边界,   \texttt{loop\_filter\_across\_titles\_enabled\_flag}的值为0,   并且a的左边界为title的边界,   \texttt{slice\_filter\_across\_titles\_enabled\_flag}的值为0,   并且a的左边界为Slice的边界)时,   将其滤波标志重置为1。
\item
标记TU的边界,   如果变换单元的划分方式为(a)图,   则将边界c的滤波标志重置为1,   如果变换单元的划分方式为(b)图,   则将b, c, d的滤波标志重置为1。
\end{enumerate}
如图\ref{LF:pic7}所示。
\begin{figure}[H]
  \centering
  \includegraphics[width=.6\textwidth]{pict/LF/DEBLOCK7.png}
  \caption{变化单元的划分方式对应的可滤波边界}
  \label{LF:pic7}
\end{figure}
标记PU的边界。如果预测单元的划分方式为(a)图和(b)图,   则将边界c的滤波标志重置为1;如果预测单元的划分方式为(c)图,   则将边界b的滤波标志重置为1;如果预测单元的划分模式为(d)图,   则将边界d的滤波标志重置为1。
如图\ref{LF:pic8}所示。
\begin{figure}[H]
  \centering
  \includegraphics{pict/LF/DEBLOCK8.png}
  \caption{预测单元的划分模式对应的可滤波边界}
  \label{LF:pic8}
\end{figure}
\subparagraph{计算边界强度}
对于第一步中滤波标志为0的边界,   其边界强度为0;滤波标志为1的边界,   则按照 $8\times 4$为基本单元计算边界强度值。如此,   对于亮度分量,   共得到32个BS值;对于色度分量,   由特定亮度分量边界强度值复制而来。
如图\ref{LF:pic9}所示。
\begin{figure}[H]
  \centering
  \includegraphics{pict/LF/DEBLOCK9.png}
  \caption{BS值获取}
  \label{LF:pic9}
\end{figure}
\subparagraph{对亮度分量进行滤波开关决策、滤波强度选择}
如图\ref{LF:pic10}所示。
\begin{figure}[H]
  \centering
  \includegraphics[width=.5\textwidth]{pict/LF/DEBLOCK10.png}
  \caption{滤波决策流程图}
  \label{LF:pic10}
\end{figure}
\subparagraph{滤波}
包括亮度分量的滤波以及色度分量的滤波,   整体上按照下列顺序进行:$a\to e \to g \to b \to c \to f \to h \d$ 。

至此,   第一个CU块的垂直边界滤波完成,   其余CU块的处理方法相同。待整幅图像的所有垂直边界滤波完成之后,   再对其进行水平边界的滤波,   前后两种滤波过程类似,   不再赘述。

在不同编码配置情况下,   去方块滤波器可以在比特率上带来不同程度的减少,   在主观感受上也有明显的改善, 其效果如图\ref{LF:deblocking}。
\begin{figure}
  \centering
  \includegraphics{pict/LF/deblocking.png}
  \caption{视频\textit{BasketballDrive},  (a)去方块关闭,  (b)去方块开启\cite{hevc}}
  \label{LF:deblocking}
\end{figure}

\subsection{样点自适应补偿}
H.265/HEVC仍采用基于块的DCT变换,   并在频域对变换系数进行量化。对于图像中的强边缘,   由于高频交流系数的量化失真,   在解码后会在边缘周围产生波纹现象,   也即振铃效应,   会给视频的主客观质量带来严重的影响,   如图\ref{LF:ringing}。造成该现象的根本原因是高频细节的丢失,   因此抑制振铃效应的关键在于减小高频分量失真,   但直接操作会带来压缩效率的降低。样点自适应补偿(SAO)技术从像素域来解决该问题,   使重构曲线中的波峰和波谷像素平滑化。
\begin{figure}
  \centering
  \subfloat[原图]{\includegraphics[width=.45\textwidth]{pict/LF/Ringing_artifact_example_-_original.png}}\hspace{2pt}
  \subfloat[重构图像]{\includegraphics[width=.45\textwidth]{pict/LF/Ringing_artifact_example.png}}
  \caption{图片显示为振铃效应产生的环状伪影。在颜色变换的两边各有3个等级:过冲,   第一环,   和(较微弱的)第二环。\url{https://en.wikipedia.org/wiki/Ringing\_artifacts}}
  \label{LF:ringing}
\end{figure}

H.265/HEVC标准中的SAO以CTB为基本单位,   通过选择一个合适的分类器将重建像素划分类别,   让后对不同类别像素使用不同补偿值,   可有效改善图像质量。包括两大类补偿形式:边界补偿(Edge Offset,   EO)和边带补偿(Band Offset,   BO),   此外还引入了参数融合技术。
\subsubsection{SAO原理简介}
\paragraph{边界补偿}
边界补偿技术是通过比较当前像素值和相邻像素值的大小对当前像素进行归类,   然后对同类像素补偿相同数值。为了均衡复杂度和编码效率,   其采用了一维三像素分类模式。

根据选取像素的位置差异,   边界补偿共分为4种模式:水平方向(EO\_0)、垂直方向(EO\_1)、 $135^{\circ}$方向(EO\_2)和 $45^{\circ}$方向(EO\_3),   如图所示,   其中,   c表示当前像素,   a和b表示相邻像素。
如图\ref{LF:pic11}所示。
\begin{figure}[H]
  \centering
  \includegraphics{pict/LF/SAO1.png}
  \caption{边界补偿模式图}
  \label{LF:pic11}
\end{figure}
在任意一种模式下,   根据以下条件将重构像素归为5个不同种类。
\begin{enumerate}
  \item 如果a<c且b<c,   则将像素c划分为种类1。
  \item 如果c<a且c=b或c<b且c=a,   则将像素c划分为种类2。
  \item 如果c>a且c=b或c>b且c=a,   则将像素c划分为种类3。
  \item 如果c>a且c>b,   则将像素c划分为种类4。
  \item 如果不属于以上4种情况,   则将像素c划分为种类0。
\end{enumerate}

种类1--4所表示的像素关系如图\ref{LF:pic12}所示,   这四个种类的边缘形状依次为谷型、凹型、凸型、峰型。
\begin{figure}
  \centering
  \includegraphics{pict/LF/SAO2.PNG}
  \caption{边界补偿分类}
  \label{LF:pic12}
\end{figure}

边界补偿技术通过先分类,   然后对于种类1--4进行补偿,   种类0不进行补偿。一般来说,   种类1--2的补偿值为正而种类3--4的补偿值为负。因此若只针对边界补偿来说,   只要传送补偿的绝对值,   便可以通过补偿种类判断其符号。

\paragraph{边带补偿}
边带补偿(BO)技术根据像素强度值进行归类,   然后等分为32条边带。例如对于8bit灰度图片,   其范围为0--255,   每条边带包含8个像素值,   之后每根边带会根据自身像素特点进行补偿,   相同边带内补偿值相同。

一般情况下,   在一定的区域内像素值的波动很小,   所以一个CTB中的大多数像素属于少数几个边带。H.265/HEVC标准规定一个CTB只能选择四条连续的边带,   并且只对着四条连续的边带中的像素进行补偿,   一方面统一了边带补偿值和边界补偿值,   另一方面减少了对存储器的要求。选择边带的原则由率失真优化方法确定,   然后将相关数据传送至解码器部分。

\paragraph{参数融合}
参数融合是针对一个CTB块,   其SAO参数直接使用相邻块的SAO参数,   这时只要标识采用了哪个相邻块的SAO参数即可。
如图\ref{LF:pic13}所示,   A、B、C均表示CTB块,   当对C块进行SAO参数决策时,   A、B块的SAO参数已经确定。此时C的SAO参数有以下三种选择:
\begin{enumerate}
  \item 直接使用A块的参数。
  \item 直接使用B块的参数。
  \item 通过分析自身像素块得到参数。
\end{enumerate}
前两种选择都属于参数融合算法,   C块仅需要传递参数融合位即可。
\begin{figure}
  \centering
  \includegraphics[width=.5\textwidth]{pict/LF/SAO3.PNG}
  \caption{参数融合}
  \label{LF:pic13}
\end{figure}

然而需要注意的是,   在使用该方法时,   一个CTU的亮度和色度分量必须具有同步的参数。也就是说,   可以同时采用相同块进行参数融合,   或者根据自身像素特征选取不同的参数。

为了直观感受SAO的效果,   不妨以实物图片为例:下图\ref{LF:pic14}中c是未经编码处理的原图片;a是经过编码处理,   但未进行SAO处理的图片;b是经过编码处理,   且进行SAO处理的图片。尽管现象不是很明显,   但是和原图c对比后可以看到,   图片a的边缘处有明显的振铃效应,   而经过SAO处理的图片b的振铃效应有了较为显著的消退,   能够更好地还原出该图片的本来风貌。

\begin{figure}
  \centering
  \includegraphics[width=.8\textwidth]{pict/LF/SAO4.PNG}
  \caption{SAO效果图}
  \label{LF:pic14}
\end{figure}
      \section{熵编码}
        上下文参考自适应二进制算术编码(Context-Based Adaptive Binary Arithmetic Coding,CABAC)是HEVC使用的熵编码方法。尽管它最早在H.264/AVC标准中提出并拥有高于绝大多数熵编码的压缩率,其数据依存性使得它难以并行处理,因此仅在Main Profile以及更高档次下可以使用。对应地在HEVC熵编码标准化过程中,编码效率以及吞吐量两方面被细致地研究了。本章节先介绍熵编码的一些概念,再给出CABAC的实现方式以及其压缩效率,HEVC中CABAC的设计准则见文献\cite{HEVC},本文并不讨论。

\subsection{熵编码概述}
由Shannon的信息论,一个离散信源X的信源熵为$H(X)=E\left[log\frac{1}{P(x_i)}\right]$,这也代表了对该信源进行可正确解码的编码后表示每个码字平均需要的最短码长,而原信源平均码长等于信源熵的充要条件是其编码后的序列服从等概率分布\cite{Shannon}。

这告诉了我们两件事:
\begin{enumerate}
  \item 我们必须以不小于原信源信源熵的空间来储存信息,额外需消耗的空间叫做这种表达方式的冗余。冗余的存在是能对信源进行压缩的前提与基础。
  \item 熵编码的实质是对离散信源进行适当的变换,使变换后新的符号序列信源尽可能为等概率分布,从而使新信源的每个码符号平均所含的信息量达到最大。这一点和密码学的表述十分相近,只不过密码学是要用少量信息来让源信源看似多了很多信息。
\end{enumerate}

实际通信中,信源通常输出的是符号序列,而符号间彼此有一定相关性,其联合熵可用来表征信源输出一个序列所提供的平均信息量,定义为$H_N(X)=H(X_1X_2\cdots X_N)$,则当信源足够长时每个符号平均信息量可表示为$h(\mathcal{X})=\lim_{N\rightarrow\infty}\frac{H_N(X)}{N}$,称为该信源的熵率。进一步来说,对于平稳Markov信源(给定当前信源符号,未来符号与过去无关)若将整个随机过程视为一整体来编码,可以证明\cite{Information_Theory}条件熵$H(X_N|X_1X_2\cdots X_{N-1})$是$N$的非增函数,则有不等式:
\begin{equation}
  h(\mathcal{X}) = \lim_{N\rightarrow\infty}\frac{H_N(X)}{N} \leq H(X_N|X_1X_2\cdots X_{N-1}) \leq H(X_i)
\end{equation}
因此自适应编码一般来说压缩率总是比非自适应算法要高。

Huffman于1952年提出了一种针对已知信源构造最优变长码的方法,被称作Huffman编码,其基本思想是为出现频率高的码字分配较短的码长,因此能够实现压缩,且有平均码长满足$H(X)\leq \bar{l} <H(X)+1$。

\subsection{算术编码} \label{EC:secAC}
算术编码也为一种熵编码方法,但由于它把整个信源当成一个符号处理,可以为单个码字分配小于1的码长,因此压缩效率更高。

算术编码原理是:根据信源概率将$[0,1)$区间划分为互不重叠的子区间,子区间宽度为各符号序列概率,这样信源符号就和各子区间一一对应。每输入一个符号就将选择的区间进一步划分,则最终该区间就对应于输入码字,实际输出的是该区间下长度最短的码字,具体编码过程见图\ref{EC:AC},其中I,K,W概率分别为0.5,0.25,0.25。
\begin{figure}
  \centering
  \includegraphics{pict/EC/Arithmetic_coding_visualisation_circle.png}
  \caption{The above example visualised as a circle, the values in red encoding "WIKI" and "KIWI" \url{https://en.wikipedia.org/wiki/Arithmetic_coding}}
  \label{EC:AC}
\end{figure}
解码过程就是根据所传输数据判断它所对应的编码区间,从而得到传输符号。

算数编码平均码长满足$H(X)\leq \bar{l} < H(x)+\frac{2}{N}$,其中$N$为序列长度。

普通的算术编码在实现上有几个问题:
\begin{enumerate}
  \item 实际计算机精度不可能无限长,因此运算过程中可能导致溢出,这点可以通过比例缩放解决。
  \item 编解码过程中需要不断进行查表并进行浮点数运算,当符号个数大的时候开销很大。\label{EC:prob}
  \item 传输过程中有一个bit出现问题则代表整个解码序列出差错,且在接收到所有序列前无法解码。
\end{enumerate}
其中针对\ref{EC:prob},本章节作者写了一个算数编/解码程序\url{github},信源符号视为1字节ASCII码256个字符,在使用了如Hash索引,二分查找等快速算法的情况下平均编码速度36MB/s,解码速度12MB/s(i7-6700单核,g++环境),效率不高。

相对来说,CABAC以二进制信源作为编码元素,并采用自适应方法编码,总体上有以下优点:
\begin{enumerate}
  \item 只编码二进制元素,因此运算复杂度较低,并且可以针对最大概率符号(Most Probable Symbol,MPS)进行概率建模。
  \item 概率模型由当前上下文自适应选择,因此可以建模的很好。
  \item 通过量化过的区间与概率状态在区间划分时不使用除法,提高了运行效率。
\end{enumerate}

\subsection{CABAC}
CABAC在编码的最后一步进行,此时视频已经变成了一系列语法元素。语法元素描述了该视频信号要怎么在解码器处重建,包括了CTU、PU、TU的结构语法,预测方式(帧内还是帧间),预测参数以及预测残差,SAO偏移等。在HEVC中也有一些语法元素并不经过CABAC编码,其他的高级语法元素采用零阶指数Golomb码或定长码来编码,具体哪些需要编码见\cite{hevc}。

\subsubsection{CABAC概览}
  CABAC编码主要包括三个基本步骤:
  \begin{enumerate}
    \item 二进制化。
    \item 上下文建模。
    \item 二进制算数编码。
  \end{enumerate}
  图\ref{EC:Block_diagram}为其具体实现框图。
  \begin{figure}
    \centering
    \includegraphics{pict/EC/Block_Diagram.png}
    \caption{CABAC block diagram (from the encoder perspective): Binarization, context modeling (including probability estimation and assignment), and binary arithmetic coding. In red: Potential throughput bottlenecks, as further discussed from the decoder perspective in Sect. 8.3.2}
    \label{EC:Block_diagram}
  \end{figure}
  二进制化指将语法元素映射为二进制符号,上下文建模预测每个常规编码符号概率,二进制算数编码将符号编码为二进制比特流。

\subsubsection{二进制化}
  CABAC编码策略是找到一种对非二进制语法元素有效的编码方案,像是运动矢量差与变换系数等可以根据先验知识设计出很有效的编码方案来预处理,之后的上下文建模与算数编码基于这个来进行。H.264/AVC与HEVC的二值化手段都是基于一些基础的可快速实现的概率模型设计的。

  HEVC二值化手段包括k阶截断Rice编码(k-th order truncated Rice,TRk),k阶指数Golomb编码(k-th order Exp-Golomb,EGk),定长二进制化(fixed-length,FL)方法,其中部分方法也备用在了H.264/AVC上。表\ref{EC:binarization}表示了用这些方法编码N的结果。
    \begin{table}
      \caption{Examples of different binarizations}
      \begin{center}
        
      \begin{tabular}{llllll}
        \hline
        & & TrU & TRk & EG & FL\\\hline
        N & Unary(U) & cMax=7 & k=1;cMax=7 & k=0 & cMax=7\\\hline
        0 & 0 & 0 & 00 & 1 & 000\\
        1 & 10 & 10 & 01 & 010 & 001\\
        2 & 110 & 110 & 100 & 011 & 010\\
        3 & 1110 & 1110 & 101 & 00100 & 011\\
        4 & 11110 & 11110 & 1100 & 00101 & 100\\
        5 & 111110 & 111110 & 1101 & 00110 & 101\\
        6 & 1111110 & 1111110 & 1110 & 00111 & 110\\
        7 & 11111110 & 1111111 & 1111 & 0001000 & 111\\\hline
      \end{tabular}
    \end{center}
    \label{EC:binarization}
    \end{table}
  \begin{itemize}
    \item 一元码编码后前N个符号为1,最后一个符号为0。当遇到0时代表编码结束。对TrU(Truncated Unary),当数值大于cMax则截断该码字。
    \item k阶截断Rice码由前缀与后缀组成:前缀为N<<k,最大值为cMax;后缀为N的最后k个符号。
    \item k阶Golomb码对于参数未知或可变的几何分布有接近最优的编码性能。每个码字由一个长为$l_N+1$的一元码字与长为$l_N+k$的后缀组成,其中$l_N=\left \lfloor{log_2((N>>k)+1)}\right \rfloor$
    \item 定长码使用长度为$\left \lceil{log_2(cMax+1)}\right \rceil $的码字表示。
  \end{itemize}

  不同的语法元素使用不同的二进制化方案,有些还与以前编码过的元素以及参数有关。大多数语法元素均采用上述方案或者其中几种的组合,还有些语法元素使用自定义的二进制化方案,具体内容参见\cite{98}。

\subsubsection{上下文建模}
  语法元素二进制化完成后,其上下文建模方法依赖于编码模式。这包括两种模式:常规编码模式与旁路编码模式。旁路编码模式不对概率模型进行自适应更新,而是假设每个符号服从均匀分布。在常规编码模式中,概率模型或根据语法元素种类,其符号索引(binIdx)等固定选择,或根据其相邻块的编码信息自适应选择。这种概率模型的选择被称为上下文建模。

  作为一项重要的设计决策,第二种情况通常只在最常见到的符号上使用,而第一种则一般来说会使用联合零阶概率模型来处理。通过这种方式,CABAC使得次符号等级的自适应概率建模变成了可能,因此也能更好地利用起符号间冗余而不需很费力地为其建模。需要注意的是,原则上由一种模型转换到另一种模型可发生在任何两个连续编码的符号中间。大体上CABAC中上下文模型的设计目的就是在避免没必要的建模消耗与充分利用符号间依赖关系中作出很好的权衡。

  CABAC中概率模型的参数也是自适应的,即是说它以输入比特为单位自适应更新其模型参数,这一过程被称为概率估计。为了这个目的,CABAC中的每个概率模型都可取126种不同的状态,其概率位于区间$[0.01875,0.98125]$。每个概率模型有两个参数,被储存为7-bit的条目(entry): 6bits用来存储63种概率状态,代表最小概率符号(LPS)的模型概率$p_{LPS}$; 1bit用来存储$v_{MPS}$,代表当前最大概率符号(MPS)。CABAC中概率估计依赖于被称为``指数退化''的模型,在$t$时刻编码完一个符号$b$后概率更新如下:
  \begin{equation}
    p_{LPS}^{(t+1)}=\begin{cases}
      \alpha*p_{LPS}^{(t)} & if ~ b = v_{MPS}\\
      1-\alpha*(1-p_{LPS}^{(t)}) & otherwise
    \end{cases}
    \label{EC:eq1}
  \end{equation}
  该式中,因子$\alpha$决定了适应速度,当$\alpha\rightarrow 1$时代表最慢的适应速度,$\alpha$越小适应速度越快。%这种方法与采用华东窗口得到的结果一致,见\cite{4,65}
  在CABAC设计中,等式\ref{EC:eq1}使用的因子$\alpha$如下:
  \begin{equation}
    \alpha = \left(\frac{0.01875}{0.5}\right)^{\frac{1}{63}} with \min_t p_{LPS}^{(t)}=0.01875
  \end{equation}
  且使用合理的量化将其分为63种状态,来得到一有限状态机(Finite-state Machine,FSM),对应一转移概率表。在HEVC中这一方式不变,尽管有其他的提案\cite{1,78}指出可以在跟高的运算复杂度下使平均比特率降低$0.8-0.9\%$。

  CABAC中每一个概率模型都使用唯一的上下文索引(ctxIdx)指代,其中该索引或由固定分配产生,或由对应上下文逻辑计算得出。在HEVC标准化过程中花费了大量精力来在吞吐量与编码效率方面改进模型分配与上下文导出逻辑。

\subsubsection{无乘法二进制算术编码}
  二进制算术编码,或者算数编码笼统来说都基于递归区间划分原理。最初给定区间由其下界(base)$L$与其宽度(range)$R$表示,接下来它被分为两个不相交子区间:一个宽度为
  \begin{equation}
    R_{LPS}=p_{LPS}*R
    \label{EC:eq2}
  \end{equation}
  的LPS区间,与其对偶的长为$R_{MPS}=R-R_{LPS}$的MPS区间。根据要编码的为LPS还是MPS,对应的子区间被选为新的编码区间,通过这样的递归划分,编码器最终输出区间$[L^{(N)},L^{(N)}+R^{(N)})$中最短的一个值作为编码结果。为了保证以有限精度表示$R^{(j)}$与$L^{(j)}$,在区间划分时需要进行重归一化操作。每当$R^{(j)}$低于一阈值就进行归一化操作,同时输出其起始无歧义比特。

  解码器处可以通过追踪区间来很快地恢复编码比特。因为CABAC只有两个符号,根据式\ref{EC:eq2}仅需比较一次编码值与边界值就可判断输入比特为0还是1。

  从具体实现角度来说,CABAC中最耗时的操作是式\ref{EC:eq2}中的乘法,如果概率预测基于缩放累积频率计数方法的话,运算过程中还会包含整数除法,这会进一步降低性能,实际上\ref{EC:secAC}中作者写的代码就是使用的这种方法。因此为了解决这个问题,实际上在H.264/AVC标准化过程中已经提出了一族无乘法二进制算数编码方法,这在之后被称为\textit{modulo coder}(M coder)\cite{43,45}。这种方法主要的特性是将基于查表的子区间划分法与上述 的FSM概率预测法结合,以及快速的旁路编码模式。
  
  \paragraph{常规编码模式}\mbox{}

  M-coder中区间划分的基本思想是将重归一化后的区间长度的可能区间长度量化为少量的K个单元。为了简化运算,一般采用均匀量化,其中$K=2^\kappa$,最终可得到集合$\textbf{W}={W_0,W_1,\cdots,W_{K-1}}$代表区间宽度。将其与LPS概率集$\textbf{P}={p_0,p_1,\cdots,p_{N-1}}$分别相乘可以得到一大小为$K\times N$的列表$\{W_k*p_n|0\leq k < K; 0 \leq n < N\}$,则可以通过选定精度来逼近式\ref{EC:eq2}所得结果。二维查找表\textit{TabRangeLPS}中的条目以概率状态索引\textit{n}与量化块索引\textit{k(R)}表示。\textit{k(R)}的计算可以采用移位运算与位掩码来进行,而掩码可被解释为取模运算,这也是M-coder名称的由来。

  在H.264/AVC中选择$\kappa=2$与$N=64$,为了使表格最大大小$2^\kappa*N\leq 256$,HEVC也采用了这种设计。若选定$\kappa=0$,二维表\textit{TabRangeLPS}退化为一维表,等价于区间长度不变。这与JBIG中的Q coder,JPEG中的QM coder,JPEG2000中的MQ coder中的做法一致,因此可以将M-coder视为更一般化的Q-coder。相比于QM/MQ coder,H.264/AVC中的M-coder的吞吐量增加了$18\%$,同时比特率降低了$2-4\%$。有趣的是吞吐量的上升主要归功与它的旁路编码模式,因为虽然查找表大小的增加可以提高编码效率,同时它也会带来吞吐量的降低。

  \paragraph{旁路编码模式}\mbox{}

  旁路编码模式中每个符号概率都被视为0.5,因此每次区间划分只需要一次比特移位,比较,以及可能有的减法便可实现。在H.264/AVC中旁路编码模式主要用在符号以及量化系数的最低有效位上,然而在HEVC中绝大多数的二进制符号都采用旁路编码模式。如前文所述,这也在很大程度上是HEVC为各个语法元素精心设计的二进制化方案的功劳,在这种情况下二进制化后的语法元素已经接近了最优前缀码。

  \paragraph{快速重归一化}\mbox{}

  任何算术编/解码吞吐量的主要瓶颈之一是重归一化过程。H.264/AVC与HEVC中该过程需要以位为单位比较判断是否需要进一步归一化,同时把结果输出到比特流,也因此降低了吞吐量。文献\cite{48}指出可以通过字节或字为单位来提高吞吐量,具体细节见\cite{47,48}。

  \paragraph{结束标识}\mbox{}

  在解码时区间细分会不断进行下去,因此为了表示解码结束需要有对应标识。在M-coder中有一个保留概率状态,对应索引为$n=63$,此时$R_{LPS}=2$。其结果是对\texttt{end\_of\_slice\_segment\_flag}, \texttt{end\_of\_sub\_stream\_one\_bit}和\texttt{pcm\_flag}等终止标识,在重归一化过程中生成7bits输出,然后再输出2bits来终止该码字。在编码器处最后写入的比特必为1,代表\texttt{rbsp\_stop\_one\_bit}。在对比特流进行封装前,二进制码字以零值填充以按字节对齐。

\subsubsection{CABAC流程}
  以下为HEVC中CABAC常规编码具体实现流程:对每一上下文索引均分配了一个初始值$initValue$,根据下式计算其状态$ucState$,包括其概率状态索引$n$与其$MPS$,$ucState=n<<1+MPS$:
  \begin{align*}
    qp & = clip3(0,51,qp)\\
    slope & = (initValue>>4)*5-45\\
    offset & = ((initValue\&15)<<3)-16\\
    initState & = clip3(1,126,(((slope * qp) >> 4) + offset))\\
    mpState & = initState>=64\\
    ucState & = ((mpState ? (initState - 64) : (63 - initState)) << 1) + mpState
  \end{align*}
  其中qp为片层量化参数,\texttt{clip3}为动态限幅函数,原型为\texttt{clip3(min,max,v)},仅当\texttt{v}在\texttt{min}与\texttt{max}间时输出\texttt{v},其余时刻输出边界值。

  概率模型自适应更新发生在每个二进制符号编码后,其更新方法为:
  \begin{verbatim}
    if( binVal = = valMps )
      pStateIdx = transIdxMps( pStateIdx )
    else {
      if( pStateIdx = = 0 )
        valMps = 1 − valMps
      pStateIdx = transIdxLps( pStateIdx )
    }
  \end{verbatim}
  状态转移表见\ref{EC:tableTrans}。
  \begin{table}
    \caption{State transition table}
    \label{EC:tableTrans}
    \begin{center}
      \begin{tabular}{|l|*{16}{c|}}
        \hline
        pStateIdx & 0 & 1 & 2 & 3 & 4 & 5 & 6 & 7 & 8 & 9 & 10 & 11 & 12 & 13 & 14 & 15\\\hline
        transIdxLps & 0 & 0 & 1 & 2 & 2 & 4 & 4 & 5 & 6 & 7 & 8 & 9 & 9 & 11 & 11 & 12\\\hline
        transIdxMps & 1 & 2 & 3 & 4 & 5 & 6 & 7 & 8 & 9 & 10 & 11 & 12 & 13 & 14 & 15 & 16\\\hline\hline
        pStateIdx & 16 & 17 & 18 & 19 & 20 & 21 & 22 & 23 & 24 & 25 & 26 & 27 & 28 & 29 & 30 & 31\\\hline
        transIdxLps & 13 & 13 & 15 & 15 & 16 & 16 & 18 & 18 & 19 & 19 & 21 & 21 & 22 & 22 & 23 & 24\\\hline
        transIdxMps & 17 & 18 & 19 & 20 & 21 & 22 & 23 & 24 & 25 & 26 & 27 & 28 & 29 & 30 & 31 & 32\\\hline\hline
        pStateIdx & 32 & 33 & 34 & 35 & 36 & 37 & 38 & 39 & 40 & 41 & 42 & 43 & 44 & 45 & 46 & 47\\\hline
        transIdxLps & 24 & 25 & 26 & 26 & 27 & 27 & 28 & 29 & 29 & 30 & 30 & 30 & 31 & 32 & 32 & 33\\\hline
        transIdxMps & 33 & 34 & 35 & 36 & 37 & 38 & 39 & 40 & 41 & 42 & 43 & 44 & 45 & 46 & 47 & 48\\\hline\hline
        pStateIdx & 48 & 49 & 50 & 51 & 52 & 53 & 54 & 55 & 56 & 57 & 58 & 59 & 60 & 61 & 62 & 63\\\hline
        transIdxLps & 33 & 33 & 34 & 34 & 35 & 35 & 35 & 36 & 36 & 36 & 37 & 37 & 37 & 38 & 38 & 63\\\hline
        transIdxMps & 49 & 50 & 51 & 52 & 53 & 54 & 55 & 56 & 57 & 58 & 59 & 60 & 61 & 62 & 62 & 63\\\hline
      \end{tabular}
    \end{center}
  \end{table}

  解码过程:
  \begin{enumerate}
    \item LPS区间ivlLpsRange更新如下:
    \begin{itemize}
      \item 由当前区间ivlCurrRange,得量化块索引qRangeIdx如下:
      
      \texttt{qRangeIdx =( ivlCurrRange >> 6 ) \& 3}
      \item 根据pStateIdx和qRangeIdx更新下界ivlLpsRange如下:
      
      \texttt{ivlLpsRange = rangeTabLps[ pStateIdx ][ qRangeIdx ]}
    \end{itemize}
    \item 首先将ivlCurrRange置为ivlCurrRange-ivlLpsRange,而后进行下一步:
    \begin{itemize}
      \item 若ivlOffset不小于ivlCurrRange,binVal置为1-valMps,ivlOffset自减ivlCurrange,ivlCurrRange置为ivlLpsRange
      \item 否则binVal置为valMps
    \end{itemize}
  \end{enumerate}
  rangeTabLps见表\ref{EC:tableRange}。
  \begin{table}
    \caption{Specification of rangeTabLps depending on the values of pStateIdx and qRangeIdx}
    \label{EC:tableRange}
    \begin{center}
      \begin{tabular}{|c|*{4}{c|}c|*{4}{c|}}\hline
        \multirow{2}{*}{pStateIdx} & \multicolumn{4}{c|}{qRangeIdx} & \multirow{2}{*}{pStateIdx} & \multicolumn{4}{c|}{qRangeIdx}\\\cline{2-5}\cline{7-10}
        & 0 & 1 & 2 & 3 & & 0 & 1 & 2 & 3\\\hline
        0 & 128 & 176 & 208 & 240 & 32 & 27 & 33 & 39 & 45\\\hline
        1 & 128 & 167 & 197 & 227 & 33 & 26 & 31 & 37 & 43\\\hline
        2 & 128 & 158 & 187 & 216 & 34 & 24 & 30 & 35 & 41\\\hline
        3 & 123 & 150 & 178 & 205 & 35 & 23 & 28 & 33 & 39\\\hline
        4 & 116 & 142 & 169 & 195 & 36 & 22 & 27 & 32 & 37\\\hline
        5 & 111 & 135 & 160 & 185 & 37 & 21 & 26 & 30 & 35\\\hline
        6 & 105 & 128 & 152 & 175 & 38 & 20 & 24 & 29 & 33\\\hline
        7 & 100 & 122 & 144 & 166 & 39 & 19 & 23 & 27 & 31\\\hline
        8 & 95 & 116 & 137 & 158 & 40 & 18 & 22 & 26 & 30\\\hline
        9 & 90 & 110 & 130 & 150 & 41 & 17 & 21 & 25 & 28\\\hline
        10 & 85 & 104 & 123 & 142 & 42 & 16 & 20 & 23 & 27\\\hline
        11 & 81 & 99 & 117 & 135 & 43 & 15 & 19 & 22 & 25\\\hline
        12 & 77 & 94 & 111 & 128 & 44 & 14 & 18 & 21 & 24\\\hline
        13 & 73 & 89 & 105 & 122 & 45 & 14 & 17 & 20 & 23\\\hline
        14 & 69 & 85 & 100 & 116 & 46 & 13 & 16 & 19 & 22\\\hline
        15 & 66 & 80 & 95 & 110 & 47 & 12 & 15 & 18 & 21\\\hline
        16 & 62 & 76 & 90 & 104 & 48 & 12 & 14 & 17 & 20\\\hline
        17 & 59 & 72 & 86 & 99 & 49 & 11 & 14 & 16 & 19\\\hline
        18 & 56 & 69 & 81 & 94 & 50 & 11 & 13 & 15 & 18\\\hline
        19 & 53 & 65 & 77 & 89 & 51 & 10 & 12 & 15 & 17\\\hline
        20 & 51 & 62 & 73 & 85 & 52 & 10 & 12 & 14 & 16\\\hline
        21 & 48 & 59 & 69 & 80 & 53 & 9 & 11 & 13 & 15\\\hline
        22 & 46 & 56 & 66 & 76 & 54 & 9 & 11 & 12 & 14\\\hline
        23 & 43 & 53 & 63 & 72 & 55 & 8 & 10 & 12 & 14\\\hline
        24 & 41 & 50 & 59 & 69 & 56 & 8 & 9 & 11 & 13\\\hline
        25 & 39 & 48 & 56 & 65 & 57 & 7 & 9 & 11 & 12\\\hline
        26 & 37 & 45 & 54 & 62 & 58 & 7 & 9 & 10 & 12\\\hline
        27 & 35 & 43 & 51 & 59 & 59 & 7 & 8 & 10 & 11\\\hline
        28 & 33 & 41 & 48 & 56 & 60 & 6 & 8 & 9 & 11\\\hline
        29 & 32 & 39 & 46 & 53 & 61 & 6 & 7 & 9 & 10\\\hline
        30 & 30 & 37 & 43 & 50 & 62 & 6 & 7 & 8 & 9\\\hline
        31 & 29 & 35 & 41 & 48 & 63 & 2 & 2 & 2 & 2\\\hline
      \end{tabular}
    \end{center}
  \end{table}
\subsubsection{运行效率}
  为了验证CABAC算法执行效率,本章作者以HM源代码为基础实现了常规编码模式下的熵编码器,托管于\url{github}。因为要体现压缩效率的话需要对应语法元素支持,这里只列出算法执行速度为8.7MB/s。而旁路编码模式只会更快,且还可以通过并行计算来实现,因此实际效率会高于该值。
%bib
      \bibliographystyle{plain}
      \bibliography{ref}
\end{document}