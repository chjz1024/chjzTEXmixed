\documentclass[a4paper, 12pt]{article}
\usepackage{ctex}
\usepackage{float}
\usepackage{graphicx}
\usepackage[bookmarks=true]{hyperref}
\begin{document}
  \tableofcontents
  \newpage
  \section{实验数据}
    \subsection{脉冲计数式鉴频器特性参数测试}
      \begin{figure}[H]
        \centering
        \includegraphics[width=\textwidth]{pict/cir1.jpg}
      \end{figure}
      \begin{figure}[H]
        \centering
          \includegraphics[width=.75\textwidth]{pict/photo_2018-12-02_19-33-07.jpg}
        \caption{无调频,   从上至下分别为TP05, 06, 07, 08}
      \end{figure}
      TP04正弦FM, 载波300kHz,   幅度100mVpp, 调制频率1kHz,   频率偏移75kHz

      可见TP05为TP04失真放大后信号, TP06对TP05反相限幅输出, TP07为TP06微分结果, TP08将TP06整形
      \begin{figure}[H]
        \centering
        \begin{minipage}[b]{.45\textwidth}
          \centering
          \includegraphics[width=\textwidth]{pict/photo_2018-12-02_19-33-18.jpg}
          \caption{FM通道1}
        \end{minipage}
        \begin{minipage}[b]{.45\textwidth}
          \centering
          \includegraphics[width=\textwidth]{pict/photo_2018-12-02_19-33-11.jpg}
          \caption{FM通道2, 3, 4}
        \end{minipage}
      \end{figure}

      如图可见,   调频现象明显,   且经鉴频器后输出波形为标准的正弦波
      \begin{figure}[H]
        \begin{minipage}[!h]{.7\textwidth}
          \centering
          \includegraphics[width=\textwidth]{pict/curve1.png}
          \caption{拟合曲线1}
        \end{minipage}
        \begin{minipage}[!h]{.2\textwidth}
          \centering
          \begin{tabular}{c|c}
            $\Delta f(kHz)$ & $u_o(mVpp)$\\
            \hline
            10&97.50\\
            20&200.5\\
            30&300.5\\
            40&407.25\\
            50&519.75\\
            60&627.75\\
            70&738.25\\
            80&844.5\\
            90&939.00\\
            100&1050
          \end{tabular}
        \end{minipage}
      \end{figure}

      $u_o$与$\Delta f$线性良好, 说明该脉冲计数鉴频器性能良好
      \[S_d = 10.644mVpp/kHz\]
    \subsection{锁相鉴频器特性参数测试}
      \begin{figure}[H]
        \centering
        \includegraphics[width=\textwidth]{pict/cir2.jpg}
      \end{figure}
      \begin{figure}[H]
        \centering
        \begin{minipage}[!h]{.7\textwidth}
          \centering
          \includegraphics[width=\textwidth]{pict/curve2.png}
          \caption{拟合曲线2}
        \end{minipage}
        \begin{minipage}[!h]{.2\textwidth}
          \centering
          \begin{tabular}{c|c}
            $\Delta f(kHz)$ & $u_o(Vpp)$\\
            \hline
            10&0.255\\
            20&0.5075\\
            30&0.75625\\
            40&1.035\\
            50&1.28\\
            60&1.5375\\
            70&1.80575\\
            80&2.05375\\
            90&2.32875\\
            100&2.58
          \end{tabular}
        \end{minipage}
      \end{figure}
      由拟合曲线, 可见$u_o$与$\Delta f$线性良好,   说明该锁相鉴频器性能良好
      \[S_d = 25.898mVpp/kHz\]
      \begin{figure}[H]
        \centering
        \begin{minipage}[!h]{.45\textwidth}
          \centering
          \includegraphics[width=\textwidth]{pict/photo_2018-12-02_19-33-20.jpg}
          \caption{$TP15(U_{FM})$}
        \end{minipage}
        \begin{minipage}[!h]{.45\textwidth}
          \centering
          \includegraphics[width=\textwidth]{pict/photo_2018-12-02_19-33-22.jpg}
          \caption{$TP18(U_o)$}
        \end{minipage}
      \end{figure}
      由上图, 输入为频率偏移较大的调频信号, $u_{FM}=1.0325Vpp, u_o=2.58Vpp, f=1kHz$  

    \subsection{电容耦合相位鉴频器特性参数测试}
      \begin{figure}[H]
        \centering
        \includegraphics[width=\textwidth]{pict/cir3.jpg}
      \end{figure}
      \subsubsection{频谱仪测量}
        \begin{figure}[H]
          \centering
          \includegraphics[width=\textwidth]{pict/photo_2018-12-02_19-33-19.jpg}          
        \end{figure}
        \begin{displaymath}
          \begin{array}{cc}
            f_{max}=9.05MHz & f_{min}=7.93MHz\\
            +\Delta f_{max}=0.55MHz & -\Delta f_{max}=-0.57MHz\\
            \multicolumn{2}{c}{2\Delta f_{max}=1.12MHz}
          \end{array}
        \end{displaymath}
      \subsubsection{示波器测量}
        \begin{figure}[H]
          \centering
          \includegraphics[width=.8\textwidth]{pict/photo_2018-12-02_19-33-23.jpg}
          \caption{平衡点, $u_o=786.5mVpp$}
        \end{figure}
        由上图, 该鉴频器输出波形为标准的正弦波
        \begin{figure}[H]
          \centering
          \begin{minipage}[!h]{.89\textwidth}
            \centering
            \includegraphics[width=\textwidth]{pict/curve3.png}
            \caption{拟合曲线3}
          \end{minipage}
          \begin{minipage}[!h]{.1\textwidth}
            \centering
            \begin{tabular}{c|c}
              $\Delta f(kHz)$ & $u_o(mVpp)$\\
              \hline
              10&105\\
              20&215.5\\
              30&322\\
              40&424.5\\
              50&525\\
              60&625.5\\
              70&728.75\\
              80&832.5\\
              90&940.75\\
              100&1025.5
            \end{tabular}
          \end{minipage}
        \end{figure}

        由于该鉴频器在很大的范围内均保持良好的线性, 可用于鉴频
        \[S_d=10.644mVpp/kHz\]
        \begin{figure}[H]
          \centering
          \begin{minipage}[!h]{.6\textwidth}
            \centering
            \includegraphics[width=\textwidth]{pict/curve4.png}
            \caption{拟合曲线4}
          \end{minipage}
          \begin{minipage}[!h]{.35\textwidth}
            \centering
            \begin{tabular}{c|cc|c}
              $f(MHz)$ & $u_o(V)$ &$f(MHz)$ & $u_o(V)$\\
              \hline
              7.0 & -0.295 & 8.6 & 0.50918\\
              7.2 & -0.40622 & 8.8 & 1.17833\\
              7.4 & -0.56626 & 9.0 & 1.27921\\
              7.6 & -0.79128 & 9.2 & 1.0892\\
              7.8 & -1.08823 & 9.4 & 0.85986\\
              8.0 & -1.3397 & 9.6 & 0.66932\\
              8.2 & -1.2328 & 9.8 & 0.52711\\
              8.4 & -0.52251 & 10.0 & 0.42352\\
              8.5 & 0.521mV\\
            \end{tabular}
          \end{minipage}
        \end{figure}

        可见该曲线与频谱仪测得曲线十分相似, 说明两种方法均可行

    \subsection{乘积型相位鉴频器特性参数测试}
      \begin{figure}[H]
        \centering
        \includegraphics[width=\textwidth]{pict/cir4.jpg}
      \end{figure}
      \subsubsection{频谱仪测量}
        \begin{figure}[H]
          \centering
          \includegraphics[width=\textwidth]{pict/photo_2018-12-02_19-33-25.jpg}
        \end{figure}
        \begin{displaymath}
          \begin{array}{cc}
            +\Delta f_{max}=192kHz & -\Delta f_{max}=--210kHz\\
            \multicolumn{2}{c}{2\Delta f_{max}=402Hz}
          \end{array}
        \end{displaymath}
      \subsubsection{示波器测量}
        \begin{figure}[H]
          \centering
          \includegraphics[width=.8\textwidth]{pict/photo_2018-12-02_19-33-26.jpg}
          \caption{平衡点, $u_o=347mVpp$}
        \end{figure}
        由上图, 该鉴频器输出波形为标准的正弦波
        \begin{figure}[H]
          \centering
          \begin{minipage}[!h]{.89\textwidth}
            \centering
            \includegraphics[width=\textwidth]{pict/curve5.png}
            \caption{拟合曲线5}
          \end{minipage}
          \begin{minipage}[!h]{.1\textwidth}
            \centering
            \begin{tabular}{c|c}
              $\Delta f(kHz)$ & $u_o(mVpp)$\\
              \hline
              10&47.5\\
              20&96.75\\
              30&143.75\\
              40&193.75\\
              50&235.5\\
              60&278\\
              70&321.25\\
              80&355.5\\
              90&387.75\\
              100&415.5
            \end{tabular}
          \end{minipage}
        \end{figure}
        该曲线明显呈线性, 但线性度不如其他曲线好, 说明这种乘积型鉴频器性能可能弱于其他类型鉴频器
        \[S_d=4.1431mVpp/kHz\]
        \begin{figure}[H]
          \centering
          \begin{minipage}[!h]{.9\textwidth}
            \centering
            \includegraphics[width=\textwidth]{pict/curve6.png}
            \caption{拟合曲线6}
          \end{minipage}
          \begin{minipage}[!h]{.9\textwidth}
            \centering
            \begin{tabular}{c|cc|c}
              $f(MHz)$ & $u_o(V)$ &$f(MHz)$ & $u_o(V)$\\
              \hline
              4.00 & 0.28492 & 4.55 & -0.114046\\
              4.05 & 0.30328 & 4.60 & -0.18582\\
              4.10 & 0.32302 & 4.65 & -0.21868\\
              4.15 & 0.34339 & 4.70 & -0.22642\\
              4.20 & 0.36204 & 4.75 & -0.22057\\
              4.25 & 0.37410 & 4.80 & -0.20840\\
              4.30 & 0.37077 & 4.85 & -0.19383\\
              4.35 & 0.33817 & 4.90 & -0.17887\\
              4.40 & 0.26208 & 4.95 & -0.16465\\
              4.45 & 0.14058 & 5.00 & -0.14961\\
              4.50 & 1.898mV
            \end{tabular}
          \end{minipage}
        \end{figure}

        该曲线与频谱仪测得曲线很相近, 且与理论曲线规律一致。中心点之外出现了较大的不对称性。考虑到实验过程中频谱仪与示波器法测量时需重新调参, 可能是频谱仪的接入电容对曲线有了一定影响

  \section{思考题}
    \begin{enumerate}
      \item 对鉴频器的性能指标要求有哪些
      \begin{enumerate}
        \item 鉴频特性曲线, 指鉴频器输出电压$u_o(t)$与输入FM信号$f$或频偏$\Delta f(t)$之间的关系曲线
        \item 鉴频中心频率, 指鉴频特性曲线原点处的频率
        \item 灵敏度, 中心频率附近单位频偏所引起的输出电压的变化两
        \item 鉴频线性范围, 鉴频特性曲线接近于直线段的频率范围
      \end{enumerate}
      \item 分析鉴频器输出波形出现失真的原因, 实验中应如何保证鉴频输出不失真
      
      由于鉴频器元件为非线性元件, 当频率范围过大超过线性区后便会出现明显的失真。为了令鉴频输出不失真, 应
    \end{enumerate}
\end{document}